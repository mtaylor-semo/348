%!TEX TS-program = lualatex
%!TEX encoding = UTF-8 Unicode

\documentclass[t]{beamer}

%%%% HANDOUTS For online Uncomment the following four lines for handout
%\documentclass[t,handout]{beamer}  %Use this for handouts.
%\usepackage{handoutWithNotes}
%\pgfpagesuselayout{3 on 1 with notes}[letterpaper,border shrink=5mm]
%	\setbeamercolor{background canvas}{bg=black!5}


%%% Including only some slides for students.
%%% Uncomment the following line. For the slides,
%%% use the labels shown below the command.
%\includeonlylecture{student}

%% For students, use \lecture{student}{student}
%% For mine, use \lecture{instructor}{instructor}


%\usepackage{pgf,pgfpages}
%\pgfpagesuselayout{4 on 1}[letterpaper,border shrink=5mm]

% FONTS
\usepackage{fontspec}
\def\mainfont{Linux Biolinum O}
\setmainfont[Ligatures={Common,TeX}, BoldFont={* Bold}, ItalicFont={* Italic}, Numbers={Proportional}]{\mainfont}
\setsansfont[Scale=MatchLowercase]{Linux Biolinum O} 
\usepackage{microtype}

\usepackage{graphicx}
	\graphicspath{%
	{/Users/mtaylor/Pictures/teach/348/lectures/}%
	{/Users/mtaylor/Pictures/teach/common/}%}%
	{img/}} % set of paths to search for images

\usepackage{amsmath,amssymb}

%\usepackage{units}

%\usepackage{booktabs}
\usepackage{multicol}
%	\setlength{\columnsep=1em}

%\usepackage{tikz}
%	\tikzstyle{every picture}+=[remember picture,overlay]

%\usepackage[version=4]{mhchem}
%\usepackage{siunitx}

%\usepackage{textcomp}
%\usepackage{setspace}
\mode<presentation>
{
  \usetheme{Lecture}
  \setbeamercovered{invisible}
  \setbeamertemplate{items}[square]
}

\usepackage{hyperref}

\newcommand\HiddenWord[1]{%
	\alt<handout>{\rule{4cm}{0.4pt}}{#1}%
}



\begin{document}

\lecture{student}{student}

{\usebackgroundtemplate{\includegraphics[width=\paperwidth]{tidal_mudflat}}
\begin{frame}[b]{Soft bottom intertidal}
\tiny\textcolor{white}{Ingolfson, Wikimedia Commons.}
\end{frame}
}
%
{\usebackgroundtemplate{\includegraphics[width=\paperwidth]{sediment_size}}
\begin{frame}[b]{Sediment size determines water retention for benthic communities.}

\tiny\textcopyright\,McGraw-Hill
\end{frame}}

%\begin{frame}[t]{Sediment size determines water retention for benthic communities.}
%
%	\includegraphics[width=\textwidth]{sediment_size}
%	
%	\vskip0pt plus 1filll
%
%	\tiny\textcopyright\,McGraw-Hill
%
%\end{frame}
%
\begin{frame}[t]{Sand drains quickly. Clay and silt retain water.}

	\includegraphics[width=\textwidth]{sediment_sorted}
	
	\vskip0pt plus 1filll

	\tiny\textcopyright\,McGraw-Hill

\end{frame}
%
\begin{frame}[t]{The soft-bottom intertidal has less obvious zonation than rocky intertidal.}

	\vspace*{-\baselineskip}

	\begin{multicols}{2}

		\begin{center}
			\includegraphics[height=0.79\textheight]{soft_bottom_lighthouse}
		\end{center}

	\columnbreak

		\begin{center}
			\includegraphics[height=0.79\textheight]{soft_bottom_estuary}
		\end{center}

	\end{multicols}

	\vspace*{-0.5\baselineskip}
	
	\vskip0pt plus 1filll

	\tiny\textcopyright\,McGraw-Hill

\end{frame}
%
\begin{frame}[t]{Horizontal zonation is less obvious because most organisms live in the substrate.}

	\includegraphics[width=0.97\textwidth]{soft_bottom_zonation}
	
	\vskip0pt plus 1filll

	\tiny\textcopyright\,McGraw-Hill

\end{frame}
%
\begin{frame}[t]{Mud flats contain diverse \highlight{infauna.}}

\vspace*{-\baselineskip}

\begin{multicols}{2}

	\begin{center}
		\includegraphics[width=0.45\textwidth]{soft_bottom_infauna1}\\
		\includegraphics[width=0.45\textwidth]{soft_bottom_infauna2}\\
	\end{center}

\columnbreak

	\hangpara Few primary producers are present.
	
	\hangpara Chemosynthetic bacteria are abundant.
	
	\hangpara \highlight{Meiofauna} fit in spaces between grains.\\
		\hspace*{1em} 0.5 mm to 0.6 µm.
		
	\hangpara Burrowing fauna cause \highlight{bioturbation.}
	
	\hangpara Zonation based on grain size.

\end{multicols}

	\vspace*{-0.5\baselineskip}
	
	\vskip0pt plus 1filll

	\tiny\textcopyright\,McGraw-Hill

\end{frame}
%
{\usebackgroundtemplate{\includegraphics[width=\paperwidth]{sediment_burrow}}
\begin{frame}[b]{The shape of the burrow enhances water flow through the burrow.}
\tiny\textcopyright\,Oxford
\end{frame}
}
%
\begin{frame}[t]{Vertical zonation depends on oxygen availability.}

\vspace*{-\baselineskip}

\begin{multicols}{2}

	\begin{center}
		\includegraphics[height=0.79\textheight]{soft_bottom_zonation_vertical}
	\end{center}

\columnbreak

	\hangpara Surface aerobic zone is oxygenated via \highlight{bioturbation.}
	
	\hangpara Deep anaerobic zone is dominated by chemosynthetic bacteria.

\end{multicols}

	\vspace*{-0.5\baselineskip}
	
	\vskip0pt plus 1filll

	\tiny\textcopyright\,McGraw-Hill

\end{frame}
%
\begin{frame}[t]{Estuaries are the interface between rivers and oceans.}

	\includegraphics[height=0.79\textheight]{pamlico_sound}\hfill
	\includegraphics[height=0.79\textheight]{soft_bottom_estuary}
	
	\vskip0pt plus 1filll

	\hfill\tiny\textcopyright\,McGraw-Hill

\end{frame}
%
\begin{frame}[t]{Five types of estuaries have been described.}
	
	\vspace*{-\baselineskip}
	
	\begin{multicols}{2}
		\begin{center}
			\includegraphics[height=0.82\textheight]{misty_fjord}
		\end{center}
	\columnbreak
	
		\hangpara\highlight{Coastal plain}
		
		\hangpara\highlight{Bar-built}
		
		\hangpara\highlight{Delta}
		
		\hangpara\highlight{Fjord}
		
		\hangpara\highlight{Tectonic}
		
	\end{multicols}

	\vskip0pt plus 1filll

	\tiny Misty Fjord, Alaska
\end{frame}
%
\begin{frame}[t]{Coastal plain estuaries are flooded or \highlight{drowned river valleys.}}
	
	\vspace*{-\baselineskip}
	
	\begin{multicols}{2}
		\begin{center}
			\includegraphics[height=0.77\textheight]{chesapeake_bay}
		\end{center}
	\columnbreak
	
		\hangpara River or coast flooded after sea level rise.
		
		\hangpara Deep $\vee$-shape.
		
		\hangpara Salinity is major abiotic factor.
		
		
	\end{multicols}

	\vskip0pt plus 1filll

	\tiny Chesapeake Bay
\end{frame}
%
\begin{frame}[t]{Bar-built estuaries have a sedimentary barrier between the lagoon and open ocean.}
	
	\vspace*{-\baselineskip}
	
	\begin{multicols}{2}
		\begin{center}
			\includegraphics[height=0.77\textheight]{pamlico_sound}
		\end{center}
	\columnbreak
	
		\hangpara Flat, shallow bottom.
		
		\hangpara Wind is major abiotic factor.
		
	\end{multicols}

	\vskip0pt plus 1filll

	\tiny Pamlico and Albemarle Sounds, NC.
\end{frame}
%
\lecture{instructor}{instructor}
%
{\usebackgroundtemplate{\includegraphics[width=\paperwidth]{bar_built_netherlands}}
\begin{frame}[b]{}
\tiny\textcolor{white}{Google Earth.}
\end{frame}
}
%
{\usebackgroundtemplate{\includegraphics[width=\paperwidth]{coastal_vs_bar}}
\begin{frame}[b]{}
\tiny\textcolor{white}{Google Earth.}
\end{frame}
}
%
\lecture{student}{student}
%
\begin{frame}[t]{Delta estuaries have shallow deposits accumulated at the mouth of the river.}
	
	\vspace*{-\baselineskip}
	
	\begin{multicols}{2}
		\begin{center}
			\includegraphics[height=0.77\textheight]{nile_delta}
		\end{center}
	\columnbreak
	
%		\hangpara Flat, shallow bottom.
		
		\hangpara Salinity is major abiotic factor.
		
	\end{multicols}

	\vskip0pt plus 1filll

	\tiny Nile River delta, Egypt.
\end{frame}
%
\lecture{instructor}{instructor}
%
{\usebackgroundtemplate{\includegraphics[width=\paperwidth]{mississippi_river_delta}}
\begin{frame}[b]{}
\tiny\textcolor{white}{NASA.}
\end{frame}
}
%
\lecture{student}{student}
%
\begin{frame}[t]{Fjords are formed by glacial scouring.}
	
	\vspace*{-\baselineskip}
	
	\begin{center}
		\includegraphics[width=0.75\textwidth]{fjord}
	\end{center}

	\vspace*{-\baselineskip}
	
	\hspace*{4em} Steep $\cup$-shape, with rock sill at seaward edge.\\
	\hspace*{4em} Tides may be major abiotic factor.


	\vskip0pt plus 1filll

	\hfill\tiny Some fjord, some place.
\end{frame}
%
\lecture{instructor}{instructor}
%
{\usebackgroundtemplate{\includegraphics[width=\paperwidth]{fjords_norway}}
\begin{frame}[b]{}
\tiny\textcolor{white}{Google Earth}
\end{frame}
}
%
\lecture{student}{student}
%
\begin{frame}[t]{Tectonic estuaries have a variable morphology.}
	
	\vspace*{-\baselineskip}
	
	\begin{center}
		\includegraphics[width=0.7\textwidth]{san_francisco_bay}
	\end{center}

	\vspace*{-\baselineskip}
	
	\hspace*{4em} The major abiotic factor varies.
	
	\vskip0pt plus 1filll

	\tiny San Francisco Bay.
\end{frame}
%
\begin{frame}[t]{}
	
	\hangpara What are the environmental parameters common to most estuaries that you think might most affect community structure?

	
	\begin{center}
		\includegraphics[width=\textwidth]{estuary}
	\end{center}

	\vskip0pt plus 1filll

	\tiny\textcopyright\,Pearson, inc.
\end{frame}
%
\begin{frame}[t]{The parameters that most affect community structure are}

	\begin{multicols}{2}

		\onslide<1->\hangpara\HiddenWord{Salinity}
	
		\onslide<2->\hangpara\HiddenWord{Temperature}
	
		\onslide<3->\hangpara\HiddenWord{Water flow}
		
		\onslide<3->\hangpara\HiddenWord{}

	\columnbreak
	
		\onslide<4->\hangpara\HiddenWord{Substrate}
		
		\onslide<5->\hangpara\HiddenWord{Turbidity \& light}
		
		\onslide<6->\hangpara\HiddenWord{}
		
		\onslide<6->\hangpara\HiddenWord{}

	\end{multicols}
	
\end{frame}
%
\begin{frame}[t]{Compare a bar-built estuary to a fjord. Consider}

	\hangpara morphology,
	
	\hangpara abiotic environment, and
	
	\hangpara vertical and horizontal zonation.

\end{frame}
%
\begin{frame}[t]{Compare a bar-built estuary to a fjord.}

	\begin{multicols}{2}

		\onslide<1->\hangpara\textbf{Bar-built}
		
		\onslide<2->\hangpara\HiddenWord{shallow and flat}
	
		\onslide<3->\hangpara\HiddenWord{wind \& wave action}
	
		\onslide<4->\hangpara\HiddenWord{turbulent and turbid}
		
		\onslide<5->\hangpara\HiddenWord{temperature variable}

		\onslide<6->\hangpara\HiddenWord{horizontally zoned}

	\columnbreak
	
		\onslide<1->\hangpara\textbf{Fjord}
		
		\onslide<2->\hangpara\HiddenWord{deep and steep}
		
		\onslide<3->\hangpara\HiddenWord{limited wave action}
		
		\onslide<4->\hangpara\HiddenWord{calm and clear}
		
		\onslide<5->\hangpara\HiddenWord{temperature stratified}
		
		\onslide<6->\hangpara\HiddenWord{vertically zoned}

	\end{multicols}
	
\end{frame}
%
{\usebackgroundtemplate{\includegraphics[width=\paperwidth]{salt_water_wedge}}
\begin{frame}[b]{Salinity distribution depends on balance between tide and freshwater flow.}
\tiny\textcopyright\,McGraw-Hill
\end{frame}
}
%
%
\begin{frame}[t]{Salinity distribution all affected by}
	
	\vspace*{-\baselineskip}
	
	\begin{multicols}{2}
		\begin{center}
			\includegraphics[height=0.79\textheight]{salinity_coriolis}
		\end{center}
	\columnbreak
	
		\hangpara Coriolis effect,
		
		\hangpara depth, and
		
		\hangpara evaporation.
		
	\end{multicols}

	\vskip0pt plus 1filll

	\tiny Benjamin-Cummings.
\end{frame}
%
\begin{frame}[t]{Salinity distribution affects organismal distribution in the estuary.}
	
	\vspace*{-\baselineskip}
	
		\begin{center}
			\includegraphics[height=0.79\textheight]{salinity_organismal_distribution}
		\end{center}

	\vskip0pt plus 1filll

	\tiny\textcopyright\,McGraw-Hill.
\end{frame}
%
\begin{frame}[t]{Osmoregulation maintains homeostasis.}

	\hangpara\highlight{Hypertonic}

	\hangpara\highlight{Hypotonic}
	
	\hangpara\highlight{Isotonic}

\end{frame}

{\usebackgroundtemplate{\includegraphics[width=\paperwidth]{gizzard_shad}}
\begin{frame}[b]{}
\hfill\tiny\textcolor{white}{Brian Gratwicke, Flickr Creative Commons.}
\end{frame}
}
%
\begin{frame}[t]{Organisms have different osmoregulation strategies.}
	
	\vspace*{-\baselineskip}
	
		\begin{center}
			\includegraphics[height=0.82\textheight]{osmoregulation}
		\end{center}

	\vskip0pt plus 1filll

	\tiny\textcopyright\,McGraw-Hill.
\end{frame}
%
\begin{frame}[t]{Sediment input affects substrate, turbidity, and light.}
	
	\vspace*{-\baselineskip}
	
	\begin{multicols}{2}
		\begin{center}
			\includegraphics[height=0.79\textheight]{louisiana_coast}
		\end{center}
	\columnbreak
	
		\hangpara Most estuaries have sediment bottom.
		
		\hangpara River flow and wave action affect turbidity.
		
		\hangpara How does turbidity affect GPP?
		
	\end{multicols}

	\vskip0pt plus 1filll

	\tiny NASA.
\end{frame}
%
{\usebackgroundtemplate{\includegraphics[width=\paperwidth]{outerbanks}}
\begin{frame}[b]{\textcolor{white}{Describe the horizontal distribution of sediments.}}
\tiny\textcolor{white}{Google Earth.}
\end{frame}
}
\end{document}
