%!TEX TS-program = lualatex
%!TEX encoding = UTF-8 Unicode

\documentclass[t]{beamer}

%%%% HANDOUTS For online Uncomment the following four lines for handout
%\documentclass[t,handout]{beamer}  %Use this for handouts.
%\usepackage{handoutWithNotes}
%\pgfpagesuselayout{3 on 1 with notes}[letterpaper,border shrink=5mm]
%	\setbeamercolor{background canvas}{bg=black!5}


%%% Including only some slides for students.
%%% Uncomment the following line. For the slides,
%%% use the labels shown below the command.
%\includeonlylecture{student}

%% For students, use \lecture{student}{student}
%% For mine, use \lecture{instructor}{instructor}


%\usepackage{pgf,pgfpages}
%\pgfpagesuselayout{4 on 1}[letterpaper,border shrink=5mm]

% FONTS
\usepackage{fontspec}
\def\mainfont{Linux Biolinum O}
\setmainfont[Ligatures={Common,TeX}, BoldFont={* Bold}, ItalicFont={* Italic}, Numbers={Proportional}]{\mainfont}
\setsansfont[Scale=MatchLowercase]{Linux Biolinum O} 
\usepackage{microtype}

\usepackage{graphicx}
	\graphicspath{%
	{/Users/mtaylor/Pictures/teach/348/lectures/}%
	{/Users/mtaylor/Pictures/teach/common/}%}%
	{img/}} % set of paths to search for images

\usepackage{amsmath,amssymb}

%\usepackage{units}

%\usepackage{booktabs}
\usepackage{multicol}
%	\setlength{\columnsep=1em}

%\usepackage{tikz}
%	\tikzstyle{every picture}+=[remember picture,overlay]

%\usepackage[version=4]{mhchem}
%\usepackage{siunitx}

%\usepackage{textcomp}
%\usepackage{setspace}
\mode<presentation>
{
  \usetheme{Lecture}
  \setbeamercovered{invisible}
  \setbeamertemplate{items}[square]
}

\usepackage{hyperref}

\newcommand\HiddenWord[1]{%
	\alt<handout>{\rule{4cm}{0.4pt}}{#1}%
}

\newcommand\skiptobottom{%
	\vskip0pt plus 1filll%
}

\begin{document}

\lecture{student}{student}

\begin{frame}[t]{What are the sources of GPP in an estuary?}

	\begin{multicols}{2}
	
		\includegraphics[width=0.45\textwidth]{mangrove_aerial_view}\\
		\includegraphics[width=0.45\textwidth]{algal_flat}
	\columnbreak
	
		\hangpara\highlight{Outwelling hypothesis}
		
		\hangpara How is NPP exported from the estuary?
		
	\end{multicols}
	
\end{frame}
%
\begin{frame}[t]{Estuaries have \highlight{microbial loops.}}
	\includegraphics[width=\textwidth]{estuary_microbial_loop}

	\skiptobottom

	\tiny\textcopyright\,Benjamin Pearson
	
\end{frame}
%
{\usebackgroundtemplate{\includegraphics[width=\paperwidth]{marsh_zonation_intro}}
\begin{frame}[b]{Marsh zonation is driven by biotic and abiotic processes.}
\hfill\tiny\textcolor{green4}{Source unknown.}
\end{frame}
}
%
\begin{frame}[t]{Marshes have \highlight{high marsh} and \highlight{low marsh} zones.}
	\includegraphics[width=\textwidth]{high_low_marsh}

	\skiptobottom

	\tiny\textcopyright\,Benjamin Pearson
	
\end{frame}
%
{\usebackgroundtemplate{\includegraphics[width=\paperwidth]{marsh_zonation_plants}}
\begin{frame}[b]{Biotic and abiotic interactions create plant zones within the low and high marshes.}
\tiny\textcopyright\,Benjamin Pearson
\end{frame}
}
%
{\usebackgroundtemplate{\includegraphics[width=\paperwidth]{marsh_zonation_abiotic}}
\begin{frame}[b]{}
\tiny Pennings and Bertness 2001. 
\end{frame}
}
%
{\usebackgroundtemplate{\includegraphics[width=\paperwidth]{marsh_patch_dynamics}}
\begin{frame}[b]{\highlight{Patch dynamics} are regulated by biotic and abiotic processes.}
\tiny Pennings and Bertness 2001. 
\end{frame}
}
%
{\usebackgroundtemplate{\includegraphics[width=\paperwidth]{marsh_positive_interactions}}
\begin{frame}[b]{Plants and animals can have \highlight{positive interactions} that promote their abundance.}
\tiny Pennings and Bertness 2001. 
\end{frame}
}
%
{\usebackgroundtemplate{\includegraphics[width=\paperwidth]{marsh_predator_prey}}
\begin{frame}[b]{Predators affect the distribution of prey in the marsh.}
\tiny\textcopyright\,Benjamin Pearson
\end{frame}
}
%
\end{document}
