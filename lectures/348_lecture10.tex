%!TEX TS-program = lualatex
%!TEX encoding = UTF-8 Unicode

%\documentclass[t]{beamer}

%%%% HANDOUTS For online Uncomment the following four lines for handout
\documentclass[t,handout]{beamer}  %Use this for handouts.
\includeonlylecture{student}
\usepackage{handoutWithNotes}
\pgfpagesuselayout{3 on 1 with notes}[letterpaper,border shrink=5mm]
%	\setbeamercolor{background canvas}{bg=black!5}


%%% Including only some slides for students.
%%% Uncomment the following line. For the slides,
%%% use the labels shown below the command.

%% For students, use \lecture{student}{student}
%% For mine, use \lecture{instructor}{instructor}


%\usepackage{pgf,pgfpages}
%\pgfpagesuselayout{4 on 1}[letterpaper,border shrink=5mm]

% FONTS
\usepackage{fontspec}
\def\mainfont{Linux Biolinum O}
\setmainfont[Ligatures={Common,TeX}, Contextuals={NoAlternate}, BoldFont={* Bold}, ItalicFont={* Italic}, Numbers={Proportional}]{\mainfont}
\setsansfont[Scale=MatchLowercase]{Linux Biolinum O} 
\usepackage{microtype}

\usepackage{graphicx}
	\graphicspath{%
	{/Users/mtaylor/Pictures/teach/348/lectures/}%
	{/Users/mtaylor/Pictures/teach/common/}%}%
	{img/}} % set of paths to search for images

\usepackage{amsmath,amssymb}

%\usepackage{units}

\usepackage{booktabs}
\usepackage{multicol}
%	\setlength{\columnsep=1em}

%\usepackage{tikz}
%	\tikzstyle{every picture}+=[remember picture,overlay]

%\usepackage{chemfig}
\usepackage[version=4]{mhchem}
\usepackage{siunitx}

\newcommand{\insertslide}[2]{%
  \makebox{\includegraphics[page=#1,width=\textwidth, trim=0in 0in 0in 0in, clip]{#2}}
}

%\usepackage{textcomp}
%\usepackage{setspace}
\mode<presentation>
{
  \usetheme{Lecture}
  \setbeamercovered{invisible}
  \setbeamertemplate{items}[square]
}

%\usepackage{animate}

\usepackage{calc} % Needed for \widthof
%\usepackage{hyperref}

\newcommand\HiddenWord[1]{%
	\alt<handout>{\rule{\widthof{#1}}{\fboxrule}}{#1}%
}



\begin{document}

\lecture{student}{student}

{
\usebackgroundtemplate{\includegraphics[width=\paperwidth]{coral_reef_intro}}
\begin{frame}[b]{\hfill\textcolor{white}{Coral Reefs}}

\tiny\hfill\textcolor{white}{James Watt, NOAA}
\end{frame}
}

\begin{frame}[t]{Do all \highlight{scleractinian} corals build reefs?}
\begin{multicols}{2}

\begin{center}
\includegraphics[width=0.4\textwidth]{coral_reef_hermatypic1} \\[1ex]

\includegraphics[width=0.4\textwidth]{coral_reef_hermatypic2} 
\end{center}

\columnbreak

	\hangpara What are the \highlight{endosymbionts?}
	
	\hangpara What other organisms contribute to reef-building?


\end{multicols}

\end{frame}


\begin{frame}[t]{Most \ce{CaCO3} in reefs is \highlight{biogenic.}}
\begin{multicols}{2}

\begin{center}
\includegraphics[width=0.4\textwidth]{calcareous_green_algae_live} \\[1ex]

\includegraphics[width=0.4\textwidth]{calcareous_green_algae_dead} 
\end{center}

\columnbreak

	\hangpara algal contributions
	
	\hangpara sediments.
	
	\vspace*{5\baselineskip}

	\includegraphics[width=0.4\textwidth]{coralline_red_algae} 

\end{multicols}

\end{frame}

{\usebackgroundtemplate{\includegraphics[width=\paperwidth]{coral_reef_distribution}}
\begin{frame}[b]{What abiotic factors determine coral reef distribution and growth?}
\tiny\textcopyright\,McGraw-Hill.
\end{frame}
}

\lecture{instructor}{instructor}
\begin{frame}[t]{What abiotic factors determine coral reef distribution and development?}
	\begin{multicols}{2}
		\includegraphics[width=0.49\textwidth]{coral_reef_distribution}
	\columnbreak
		\hangpara Temperature \pause
		
		\hangpara Light \pause
		
		\hangpara Turbidity / sedimentation \pause
		
		\hangpara Salinity \pause
		
		\hangpara Water flow
	\end{multicols}
\end{frame}


\lecture{student}{student}
{\usebackgroundtemplate{\includegraphics[width=\paperwidth]{florida_bay}}
\begin{frame}[b]

\tiny\textcolor{white}{Florida Bay by Jeff Schmaltz, NASA.}
\end{frame}
}


\begin{frame}[t]{What are the nutrient sources of reef GPP?}
	\begin{multicols}{2}
		\begin{center}
		\includegraphics[width=0.45\textwidth]{reef_clear_water}
		
		\vspace*{1ex}
		
		\includegraphics[width=0.45\textwidth]{coral_reef_hermatypic2}
		\end{center}
		
	\columnbreak
	
		\hangpara\parbox[t]{1.5in}{\raggedright Why does water around the reef have low NPP from reef?}
		
		\hangpara Reef: 1500–5000 \si{\gram~C.m^{-2}.yr^{-1}}
		
		\hangpara Water: 18–50 \si{\gram~C.m^{-2}.yr^{-1}}

	\end{multicols}
\end{frame}


\lecture{instructor}{instructor}

\begin{frame}[t]{What are the nutrient sources of reef GPP?}
	\begin{multicols}{2}
		\begin{center}
		\includegraphics[width=0.45\textwidth]{reef_clear_water}
		
		\vspace*{1ex}
		
		\includegraphics[width=0.45\textwidth]{coral_reef_hermatypic2}
		\end{center}
		
	\columnbreak
	
		\hangpara Zooxanthellae \pause
		
		\hangpara Algae
		
		\hangpara Turf algae \pause
		
		\hangpara Bacteria
		
	\end{multicols}
\end{frame}



\lecture{student}{student}

\begin{frame}[t]{Insular coral reef formation has three stages, based on \highlight{subsidence.}}

\includegraphics[width=\textwidth]{insular_reef_formation}

\vskip0pt plus 1filll

\tiny\textcopyright\,McGraw-Hill
\end{frame}

{\usebackgroundtemplate{\includegraphics[width=\paperwidth]{continental_reef}}
\begin{frame}[t]


\vspace*{2\baselineskip}

\hangpara\hspace*{60mm}\parbox[t]{1.5in}{Continental reefs do not have an atoll stage.}

\vspace*{\baselineskip}

\hangpara\hspace*{60mm}\parbox[t]{1.5in}{Why not?}

\vskip0pt plus 1filll

\hfill\tiny Great Barrier Reef, NASA.
\end{frame}
}


\begin{frame}[t]{Fringing reef.}

\includegraphics[width=\textwidth]{reef_fringing}

\vskip0pt plus 1filll

\tiny\textcopyright\,McGraw-Hill
\end{frame}


{\usebackgroundtemplate{\includegraphics[width=\paperwidth]{reef_fringing_example}}
\begin{frame}[b]{\hfill\textcolor{white}{Fringing reef, Yap}}

\hfill\tiny\textcolor{white}{NOAA, public domain.}
\end{frame}
}

{\usebackgroundtemplate{\includegraphics[width=\paperwidth]{reef_barrier}}
\begin{frame}[b]{Barrier reef.}

\tiny\hfill\textcopyright\,McGraw-Hill
\end{frame}
}

{\usebackgroundtemplate{\includegraphics[width=\paperwidth]{reef_barrier_example}}
\begin{frame}[b]{\hfill\textcolor{white}{Barrier reef, \\\hfill New Caledonia.}}

\tiny\textcolor{white}{NASA Earth Observatory, public domain.}
\end{frame}
}

\begin{frame}[t]{Atoll.}

\includegraphics[width=\textwidth]{reef_atoll}

\vskip0pt plus 1filll

\tiny\textcopyright\,McGraw-Hill
\end{frame}

{\usebackgroundtemplate{\includegraphics[width=\paperwidth]{reef_atoll_example}}
\begin{frame}[b]{\hfill\textcolor{white}{Atafu Atoll, \\\hfill Tokelau.}}

\tiny\textcolor{white}{NASA Johnson Space Center, public domain.}
\end{frame}
}

\begin{frame}[t]{Coral reefs show distinct zonation.}

\includegraphics[width=\textwidth]{reef_zonation}

\vskip0pt plus 1filll

\tiny\textcopyright\,Benjamin-Cummings
\end{frame}

\begin{frame}[t]{What abiotic and biotic factors determine reef zonation?}

\includegraphics[width=\textwidth]{reef_zonation_crest}

\vskip0pt plus 1filll

\tiny\textcopyright\,Benjamin-Cummings
\end{frame}

\begin{frame}[t]{What factors are present in the reef crest?}

\begin{multicols}{2}
	\begin{center}
	
	\includegraphics[width=0.4\textwidth]{reef_crest1}\\[1ex]
	
	\includegraphics[width=0.4\textwidth]{reef_crest2}
	\end{center}
\columnbreak

	\hangpara\textbf{Abiotic}
	
	\vspace*{4\baselineskip}
	
	\hangpara\textbf{Biotic}
	
	
\end{multicols}
\end{frame}

\lecture{instructor}{instructor}
\begin{frame}[t]{What factors are present in the reef crest?}

\begin{multicols}{2}
	\begin{center}
	
	\includegraphics[width=0.4\textwidth]{reef_crest1}\\[1ex]
	
	\includegraphics[width=0.4\textwidth]{reef_crest2}
	\end{center}
\columnbreak

	\hangpara\textbf{Abiotic}
	
	\hangpara wave action\\\pause intense light\\\pause aerial exposure
	
	\hangpara\textbf{Biotic}
	
	\hangpara competition is high\\\pause predation is low\\\pause mutualism is low
	
\end{multicols}
\end{frame}



\lecture{student}{student}

\begin{frame}[t]{What abiotic and biotic factors determine reef zonation?}

\includegraphics[width=\textwidth]{reef_zonation_flat}

\vskip0pt plus 1filll

\tiny\textcopyright\,Benjamin-Cummings
\end{frame}

\begin{frame}[t]{What factors are present in the reef flat?}

\begin{multicols}{2}
	\begin{center}
	
	\includegraphics[width=0.4\textwidth]{reef_flat1}\\[1ex]
	
	\includegraphics[width=0.4\textwidth]{reef_flat2}
	\end{center}
\columnbreak

	\hangpara\textbf{Abiotic}
	
	\vspace*{4\baselineskip}
	
	\hangpara\textbf{Biotic}
	
	
\end{multicols}
\end{frame}

\lecture{instructor}{instructor}
\begin{frame}[t]{What factors are present in the reef flat?}

\begin{multicols}{2}
	\begin{center}
	
	\includegraphics[width=0.4\textwidth]{reef_flat1}\\[1ex]
	
	\includegraphics[width=0.4\textwidth]{reef_flat2}
	\end{center}
\columnbreak

	\hangpara\textbf{Abiotic}
	
	\hangpara intense light\\\pause salinity\\\pause aerial exposure\\\pause turbidity

	\hangpara\textbf{Biotic}
	
	\hangpara competition is high\\\pause predation is moderate\\\pause mutualism is moderate.

	
\end{multicols}
\end{frame}


\lecture{student}{student}

\begin{frame}[t]{What abiotic and biotic factors determine reef zonation?}

\includegraphics[width=\textwidth]{reef_zonation_slope}

\vskip0pt plus 1filll

\tiny\textcopyright\,Benjamin-Cummings
\end{frame}

\begin{frame}[t]{What factors are present in the fore-reef slope?}

\begin{multicols}{2}
	\begin{center}
	
	\includegraphics[width=0.4\textwidth]{reef_slope1}\\[1ex]
	
	\includegraphics[width=0.4\textwidth]{reef_slope2}
	\end{center}
\columnbreak

	\hangpara\textbf{Abiotic}
	
	\vspace*{4\baselineskip}
	
	\hangpara\textbf{Biotic}
	
	
\end{multicols}
\end{frame}

\lecture{instructor}{instructor}
\begin{frame}[t]{What factors are present in the fore-reef slope?}

\begin{multicols}{2}
	\begin{center}
	
	\includegraphics[width=0.4\textwidth]{reef_slope1}\\[1ex]
	
	\includegraphics[width=0.4\textwidth]{reef_slope2}
	\end{center}
\columnbreak

	\hangpara\textbf{Abiotic}
	
	\hangpara intense light\pause 

	\hangpara\textbf{Biotic}
	
	\hangpara competition is high\\\pause predation is high\\\pause mutualism is high.

	
\end{multicols}
\end{frame}


\begin{frame}[t]{Species have competitive adaptations.}

\begin{multicols}{2}
	\begin{center}
	
	\includegraphics[width=0.4\textwidth]{reef_slope2}\\[1ex]
	
	\includegraphics[width=0.4\textwidth]{coral_overgrowth}
	\end{center}
\columnbreak

	\hangpara\textbf{Sessile organisms}
	
	\hangpara interference (\href{https://www.youtube.com/watch?v=XsNsWTlFcCE}{video})\\\pause exploitative\\\pause chemical compounds\\\pause overgrowth

	\hangpara\textbf{Mobile organisms}
	
	\hangpara poorly understood\\\pause probably complex.

	
\end{multicols}
\end{frame}

\lecture{student}{student}
{\usebackgroundtemplate{\includegraphics[width=\paperwidth]{competition_stegastes}}
\begin{frame}[b]

\tiny\hfill Robertson 1996. Ecology 77: 885.
\end{frame}
}


\begin{frame}

\begin{multicols}{2}

	\includegraphics[width=0.49\textwidth]{seafan_predation}

\columnbreak

	\hangpara Micropredation
	
	\hangpara Herbivory
	
	\hangpara Defense mechanisms
	
	\vspace*{\baselineskip}
	
	\includegraphics[width=0.45\textwidth]{crown_thorns}
	
\end{multicols}

\end{frame}



\begin{frame}[t]{Three types of \highlight{positive interactions} are common on reefs.}

\begin{multicols}{2}

	\begin{center}
	
	\includegraphics[width=0.45\textwidth]{sponge_goby}\\[1ex]
	
	\includegraphics[width=0.45\textwidth]{cleaner_shrimp}
	
	\end{center}
	
\columnbreak

	\hangpara Indirect
	
	\hangpara Commensalism
	
	\hangpara Mutualism
		
\end{multicols}

\end{frame}

\lecture{instructor}{instructor}
{\usebackgroundtemplate{\includegraphics[width=\paperwidth]{mutualism_cleaner_wrasse}}
\begin{frame}[b]

\tiny\textcolor{white}{Klaus Stiefel, Flickr, Creative Commons.}
\end{frame}
}

{\usebackgroundtemplate{\includegraphics[width=\paperwidth]{mutualism_cleaner_shrimp}}
\begin{frame}[b]

\tiny\hfill\textcolor{white}{Scanned from a dive magazine back in the day.}
\end{frame}
}

\lecture{student}{student}
\begin{frame}[t]{Why is reef diversity highest in the Indo-West Pacific?}

\vspace*{-\baselineskip}

\begin{multicols}{2}
	\qquad\qquad\textbf{Pacific}\\
	\qquad\qquad 500 coral spp.\\
	\qquad\qquad 5000 mollusk spp.\\
	\qquad\qquad 2000 fish spp.
	
\columnbreak
	
	\qquad\textbf{Atlantic}\\
	\qquad62 coral spp.\\
	\qquad 1200 mollusk spp.\\
	\qquad 600 fish spp.
\end{multicols}

\includegraphics[width=\textwidth]{reef_diversity_distribution}

\hangpara What are the \highlight{biological} implications of high diversity?

\vskip0pt plus 1filll

\tiny\textcopyright\,Benjamin-Cummings
\end{frame}

{\usebackgroundtemplate{\includegraphics[width=\paperwidth]{reef_community_structure}}
\begin{frame}[b]{\textcolor{white}{Community structure.}}
\tiny\textcolor{white}{Elphine Stone Reef, Red Sea, Egypt. Derek Keats, Flickr, Creative Commons.}
\end{frame}
}

%% April Fool

\begin{frame}[t]{\reflectbox{Community structure is complex on coral reefs.}}

\begin{multicols}{2}
	{\centering
	\includegraphics[width=0.43\textwidth]{reef_fish_community}\par}
\columnbreak
	
	\hangpara\reflectbox{\highlight{Large numbers} of species.}
	
	\hangpara\reflectbox{\highlight{High abundance} of most species.}
	
	\hangpara\reflectbox{\highlight{High specialization} of species.}
	
	\hangpara\reflectbox{\highlight{High reproductive output}}
	
	\hangpara\reflectbox{\highlight{High variation} of community across}\newline \reflectbox{space and time.}
	
	\hangpara \reflectbox{What is your hypothesis to explain} \reflectbox{structure?}
\end{multicols}

\end{frame}

%%

\begin{frame}[t]{Community structure is complex on coral reefs.}

\begin{multicols}{2}
	{\centering
	\includegraphics[width=0.43\textwidth]{reef_fish_community}\par}
\columnbreak
	
	\hangpara\highlight{Large numbers} of species.
	
	\hangpara\highlight{High abundance} of most species.
	
	\hangpara\highlight{High specialization} of species.
	
	\hangpara\highlight{High reproductive output}
	
	\hangpara\highlight{High variation} of community across space and time.
	
	\hangpara What is your hypothesis to explain structure?
\end{multicols}

\end{frame}

{\usebackgroundtemplate{\includegraphics[width=\paperwidth]{reef_community_structure_hypotheses}}
\begin{frame}[b]{}
\end{frame}
}


\begin{frame}
\vspace{-\baselineskip}
\insertslide{6}{478_Lecture09_instructor.pdf}
\end{frame}

\begin{frame}
\vspace{-\baselineskip}
\insertslide{7}{478_Lecture09_instructor.pdf}
\end{frame}

\begin{frame}
\vspace{-\baselineskip}
\insertslide{9}{478_Lecture09_instructor.pdf}
\end{frame}

\label{key}\begin{frame}
\vspace{-\baselineskip}
\insertslide{8}{478_Lecture09_instructor.pdf}
\end{frame}

\end{document}
