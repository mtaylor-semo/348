%!TEX TS-program = lualatex
%!TEX encoding = UTF-8 Unicode

\documentclass[t]{beamer}

%%%% HANDOUTS For online Uncomment the following four lines for handout
%\documentclass[t,handout]{beamer}  %Use this for handouts.
%\usepackage{handoutWithNotes}
%\pgfpagesuselayout{3 on 1 with notes}[letterpaper,border shrink=5mm]
%	\setbeamercolor{background canvas}{bg=black!5}


%%% Including only some slides for students.
%%% Uncomment the following line. For the slides,
%%% use the labels shown below the command.
%\includeonlylecture{student}

%% For students, use \lecture{student}{student}
%% For mine, use \lecture{instructor}{instructor}


%\usepackage{pgf,pgfpages}
%\pgfpagesuselayout{4 on 1}[letterpaper,border shrink=5mm]

% FONTS
\usepackage{fontspec}
\def\mainfont{Linux Biolinum O}
\setmainfont[Ligatures={Common,TeX}, Contextuals={NoAlternate}, BoldFont={* Bold}, ItalicFont={* Italic}, Numbers={Proportional}]{\mainfont}
\setsansfont[Scale=MatchLowercase]{Linux Biolinum O} 
\usepackage{microtype}

\usepackage{graphicx}
	\graphicspath{%
	{/Users/goby/Pictures/teach/348/lectures/}%
	{/Users/goby/Pictures/teach/common/}%}%
	{img/}} % set of paths to search for images

\usepackage{amsmath,amssymb}

%\usepackage{units}

\usepackage{booktabs}
\usepackage{multicol}
%	\setlength{\columnsep=1em}

%\usepackage{tikz}
%	\tikzstyle{every picture}+=[remember picture,overlay]

%\usepackage{chemfig}
%\usepackage[version=4]{mhchem}

%\usepackage{textcomp}
%\usepackage{setspace}
\mode<presentation>
{
  \usetheme{Lecture}
  \setbeamercovered{invisible}
  \setbeamertemplate{items}[square]
}

%\usepackage{animate}

\usepackage{calc} % Needed for \widthof
%\usepackage{hyperref}

\newcommand\HiddenWord[1]{%
	\alt<handout>{\rule{\widthof{#1}}{\fboxrule}}{#1}%
}



\begin{document}

\lecture{student}{student}

{\usebackgroundtemplate{\includegraphics[width=\paperwidth]{global_continental_shelf}}
\begin{frame}[b]{\textcolor{white}{Sublittoral ecosystems.}}
\tiny\textcolor{white}{Seafloor Topography V4.0, \textcopyright\,W.H.F. Smith \& D.T. Sandwell, 1996.}
\end{frame}
}

\begin{frame}[t]{}
\includegraphics[width=\textwidth]{wave_features}

\vskip0pt plus 1filll

\tiny\textcopyright\,McGraw-Hill
\end{frame}


\begin{frame}[t]{Energy moves through the water. Waves do not move.}
\includegraphics[width=\textwidth]{wave_motion}

\vskip0pt plus 1filll

\tiny\textcopyright\,McGraw-Hill
\end{frame}

\begin{frame}[t]{Wind and shore alter wave shape.}
\includegraphics[width=\textwidth]{wave_shape}

\vskip0pt plus 1filll

\tiny\textcopyright\,McGraw-Hill
\end{frame}

{\usebackgroundtemplate{\includegraphics[width=\paperwidth]{surfer_joe}}
\begin{frame}[b]

\tiny\textcolor{white}{2010 Mavericks Competition photo by Shalom Jacobovitz, Wikimedia Commons.}
\end{frame}
}

\begin{frame}[t]{\highlight{Rogue waves} are atypically large.}
\vspace*{-\baselineskip}
\begin{multicols}{2}

	{\centering\includegraphics[width=0.49\textwidth]{rogue_wave_graph}

	\includegraphics[width=0.45\textwidth]{rogue_wave_image}\par}

\columnbreak
	\hangpara At least 2$\times$ mean wave height.
	
	\hangpara 30 m rogues have been reported.
	
\end{multicols}
\end{frame}

{\usebackgroundtemplate{\includegraphics[width=\paperwidth]{tsunami_indonesia}}
\begin{frame}[b]{\highlight{Tsunami} are high-energy, long wavelength seismic waves.}

\hfill\tiny\textcopyright\,McGraw-Hill.
\end{frame}
}


{\usebackgroundtemplate{\includegraphics[width=\paperwidth]{tsunami_tohoku}}
\begin{frame}[b]

\hfill\tiny NOAA Center for Tsunami Research.
\end{frame}
}



{\usebackgroundtemplate{\includegraphics[width=\paperwidth]{subtidal_seal}}
\begin{frame}[b]

\hfill\tiny\textcolor{white}{Sue Scott, Darwin Initiative, Flickr Creative Commons.}
\end{frame}
}

{\usebackgroundtemplate{\includegraphics[width=\paperwidth]{abiotic_gradients}}
\begin{frame}[t]{Abiotic gradients vary shallow to deep.}

%\hfill\tiny\textcolor{white}{Sue Scott, Darwin Initiative, Flickr Creative Commons.}
\end{frame}
}

{\usebackgroundtemplate{\includegraphics[width=\paperwidth]{sublittoral_divisions}}
\begin{frame}[t]{Light penetration divides sublittoral into \highlight{infralittoral} and \highlight{circalittoral} zones.}

%\hfill\tiny\textcolor{white}{Sue Scott, Darwin Initiative, Flickr Creative Commons.}
\end{frame}
}

\begin{frame}[t]{Biotic gradients also vary with depth.}
\begin{multicols}{2}
	\includegraphics[height=0.8\textheight]{biotic_gradients}
\columnbreak
	\hangpara Larval recruitment is variable.
	
	\hangpara Competition for space is intense.
	
	\hangpara Primary consumer is sea urchins.
	
\end{multicols}
\end{frame}



{\usebackgroundtemplate{\includegraphics[width=\paperwidth]{subtidal_diversity_hypothesis}}
\begin{frame}[t]{Does light fully explain this pattern?}

%\hfill\tiny\textcolor{white}{Sue Scott, Darwin Initiative, Flickr Creative Commons.}
\end{frame}
}

{\usebackgroundtemplate{\includegraphics[width=\paperwidth]{subtidal_urchins_seastars}}
\begin{frame}[t]

%\hfill\tiny\textcolor{white}{Sue Scott, Darwin Initiative, Flickr Creative Commons.}
\end{frame}
}

{\usebackgroundtemplate{\includegraphics[width=\paperwidth]{subtidal_trophic_cascade}}
\begin{frame}[t]

%\hfill\tiny\textcolor{white}{Sue Scott, Darwin Initiative, Flickr Creative Commons.}
\end{frame}
}

{\usebackgroundtemplate{\includegraphics[width=\paperwidth]{subtidal_alternate_stable_states}}
\begin{frame}[t]{Kelp-urchin interactions result in \highlight{alternate stable states.}}

%\hfill\tiny\textcolor{white}{Sue Scott, Darwin Initiative, Flickr Creative Commons.}
\end{frame}
}

\end{document}
