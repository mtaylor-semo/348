%!TEX TS-program = lualatex
%!TEX encoding = UTF-8 Unicode

\documentclass[t]{beamer}

%%%% HANDOUTS For online Uncomment the following four lines for handout
%\documentclass[t,handout]{beamer}  %Use this for handouts.
%\usepackage{handoutWithNotes}
%\pgfpagesuselayout{3 on 1 with notes}[letterpaper,border shrink=5mm]

%%% Including only some slides for students.
%%% Uncomment the following line. For the slides,
%%% use the labels shown below the command.
%\includeonlylecture{student}

%% For students, use \lecture{student}{student}
%% For mine, use \lecture{instructor}{instructor}


%\usepackage{pgf,pgfpages}
%\pgfpagesuselayout{4 on 1}[letterpaper,border shrink=5mm]

% FONTS
\usepackage{fontspec}
\def\mainfont{Linux Biolinum O}
\setmainfont[Ligatures={Common,TeX}, Contextuals={NoAlternate}, BoldFont={* Bold}, ItalicFont={* Italic}, Numbers={Proportional}]{\mainfont}
\setsansfont[Scale=MatchLowercase]{Linux Biolinum O} 
\usepackage{microtype}

\usepackage{graphicx}
	\graphicspath{%
	{/Users/goby/Pictures/teach/348/lectures/}%
	{/Users/goby/Pictures/teach/common/}%}%
	{img/}} % set of paths to search for images

\usepackage{amsmath,amssymb}

%\usepackage{units}

\usepackage{booktabs}
\usepackage{multicol}
%	\setlength{\columnsep=1em}

%\usepackage{textcomp}
%\usepackage{setspace}
\usepackage{tikz}
	\tikzstyle{every picture}+=[remember picture,overlay]

\mode<presentation>
{
  \usetheme{Lecture}
  \setbeamercovered{invisible}
  \setbeamertemplate{items}[square]
}

\usepackage{calc} % Needed for \widthof
\usepackage{hyperref}

\newcommand\HiddenWord[1]{%
	\alt<handout>{\rule{\widthof{#1}}{\fboxrule}}{#1}%
}



\begin{document}

%\lecture{instructor}{instructor}
%{\usebackgroundtemplate{\includegraphics[width=\paperwidth]{alaska_earthquakes}}
%\begin{frame}[b]{Alaska's 7.1 magnitude earthquake, 24 Jan 2016.}
%	\begin{tikzpicture}[quake/.style={circle, inner sep=0pt, minimum size=4mm}]
%		\node[quake] (alaska) at (7.6,4.1) [fill=orange5] {};
%	\end{tikzpicture}
%\end{frame}
%}

\lecture{student}{student}

\begin{frame}[t]%{What are the salts in salt water?}
	\begin{center}
	\includegraphics[height=0.9\textheight]{salinity_breakdown}
	\end{center}

\vskip0pt plus 1filll

\tiny \textcopyright\,McGraw-Hill
\end{frame}


\begin{frame}[t]{Salts have terrestrial origins.}
	\begin{center}
	\includegraphics[height=0.8\textheight]{salinity_sources}
	\end{center}

\vskip0pt plus 1filll

\tiny \textcopyright\,McGraw-Hill
\end{frame}

\begin{frame}[t]{Great Salt Lake has a mean salinity of 27\% (270\text{\textperthousand}).}
	\begin{center}
		\includegraphics[height=0.8\textheight]{salinity_great_salt_lake}
	\end{center}

	\tiny Image: Public domain, Wikimedia Commons.
\end{frame}

\begin{frame}[t]{Oceans are divided into horizontal and vertical zones.}
	\begin{center}
	\includegraphics[height=0.8\textheight]{ocean_zones}
	\end{center}

\vskip0pt plus 1filll

\tiny \textcopyright\,Benjamin-Cummins
\end{frame}

\begin{frame}[t]{Ocean circulation and epipelagic organisms.}
	\begin{center}
	\includegraphics[width=\textwidth]{ocean_circulation_intro}
	\end{center}

\vskip0pt plus 1filll

%\tiny \textcopyright\,Benjamin-Cummins
\end{frame}

\begin{frame}[t]{Solar input at equator causes warm air mass to rise.}
	\begin{center}
	\includegraphics[width=0.89\textwidth]{circulation_solar_input}
	\end{center}

\vskip0pt plus 1filll

\tiny \textcopyright\,McGraw-Hill
\end{frame}

\begin{frame}[t]{Cooler air moves along surface to replace rising air.}
	\begin{center}
	\includegraphics[height=0.8\textheight]{circulation_hadley_cells}
	\end{center}

\vskip0pt plus 1filll

\tiny \textcopyright\,McGraw-Hill
\end{frame}


\begin{frame}[t]{The \highlight{Coriolis effect} causes winds and currents to bend.}
	\begin{center}
	\includegraphics[width=\textwidth]{circulation_coriolis_effect}
	\end{center}

\vskip0pt plus 1filll

\tiny \textcopyright\,McGraw-Hill \hfill \href{https://youtu.be/aeY9tY9vKgs}{Link to Video}%\href{https://www.youtube.com/watch?v=_36MiCUS1ro}{Link to Video}
\end{frame}

\begin{frame}[t]{\highlight{Ekman transport} causes surface currents to move 90$^\circ$ relative to wind direction.}
	\begin{multicols}{2}

		\includegraphics[width=0.49\textwidth]{circulation_ekman_transport}

	\columnbreak

		\hangpara\parbox[t]{0.45\textwidth}{Currents move to right in northern hemisphere and to the left in the southern hemisphere.}
	
%		\vspace*{\baselineskip}
	
%		\hangpara\parbox[t]{0.45\textwidth}{.}
	
	\end{multicols}

	\vskip0pt plus 1filll

	\tiny \textcopyright\,McGraw-Hill

\end{frame}

{\usebackgroundtemplate{\includegraphics[width=\paperwidth]{circulation_ocean_gyres}}
\begin{frame}[b]{Surface currents form large \highlight{gyres} in each ocean.}
\tiny \textcopyright\,McGraw-Hill
\end{frame}
}

{\usebackgroundtemplate{\includegraphics[width=\paperwidth]{circulation_temperature_distribution}}
\begin{frame}[b]{Surface gyres cause uneven distribution of surface temperatures.}
\tiny \hfill\textcopyright\,McGraw-Hill
\end{frame}
}

\begin{frame}[t]{Why does circulation matter to marine organisms?}
	\hangpara List reasons why surface circulation matters to marine organisms.

	\hangpara List some types of organisms that depend on surface circulation.
\end{frame}

\begin{frame}[t]{The \highlight{Reynolds number} reflects the ease of an organism to move through a fluid medium like sea water.}

\begin{multicols}{2}
	{\LARGE
		\[Re = \frac{Vl\rho}{\mu}\]
	}

	\hangpara $V$ = velocity,\\
	$l$ = length, and\\
	$\rho$ and $\mu$ are constants.

\columnbreak

	\hangpara Will $Re$ be larger or smaller if the:

		\hangpara\hspace*{1em}  organism is long? \pause\HiddenWord{larger}

		\hangpara\hspace*{1em}  organism is short? \pause\HiddenWord{smaller}

\end{multicols}
\end{frame}



\begin{frame}[t]{\highlight{\textit{Re}} affects swimming and feeding.}
\begin{multicols}{2}

	\includegraphics[width=0.45\textwidth]{reynolds_whale_shark}

	Organisms with high \textit{Re} move \pause\HiddenWord{quickly}.

\columnbreak

	\includegraphics[width=0.45\textwidth]{reynolds_copepod}

	Organisms with low \textit{Re} move \pause\HiddenWord{slowly}.

\end{multicols}
\vskip0pt plus 1filll

\href{http://www.youtube.com/watch?v=_B8qiqeDrI0}{Link to Video} \hfill \href{http://www.youtube.com/watch?v=YaVCSzWYeAw}{Link to Video}
\end{frame}


\begin{frame}[t]{\highlight{\textit{Re}} affects body shape.}
	\includegraphics[width=\textwidth]{reynolds_streamlining}
	\begin{multicols}{2}
		\hangpara Larger, faster organisms (high \textit{Re}) tend to be streamlined, reducing drag.

		\columnbreak

		\hangpara Smaller, slower organisms (low \textit{Re}) are less streamlined, increasing drag.

	\end{multicols}

\vskip0pt plus 1filll

\tiny Image \textcopyright\, Levinton, Oxford Univ. Press.
\end{frame}

{\usebackgroundtemplate{\includegraphics[width=\paperwidth]{planktonic_community}}
\begin{frame}[b]
\hspace*{60mm}\tiny\textcolor{white}{Images: Wikimedia Commons}
\end{frame}
}

\begin{frame}[t]{Most organisms have \highlight{planktonic larvae.}}
	\begin{multicols}{2}

	\begin{center}
		\includegraphics[width=95px]{larvae_nonpelagic}
		
		\includegraphics[width=95px]{larvae_lecithotrophic}

		\includegraphics[width=95px]{larvae_planktotrophic}
	\end{center}	
	\columnbreak
	
		Non-pelagic
		
		Lecithotrophic
		
		Planktotrophic
		
		\vspace*{\baselineskip}
		
		Which can disperse the farthest? Why?		
		
		\vspace*{\baselineskip}
		
		Which has the highest mortality? Why?
	
	\end{multicols}
	
	\vskip0pt plus 1filll
	
	\tiny T-B: \textcopyright\,M. Hellberg; M. Perkins, Flickr CC; NOAA
\end{frame}

\begin{frame}[t]{Which species produces more offspring and why?}
	\begin{multicols}{2}

	\begin{center}
		\includegraphics[height=0.45\textheight]{larval_strategies_lecitho}
		
		\textit{Lithodes couesi}\\
		lecithotrophic

	\end{center}	
	\columnbreak

	\begin{center}	
		\includegraphics[height=0.45\textheight]{larval_strategies_plankto}

		\textit{Macroregonia macrochira}\\
		planktotrophic
	\end{center}
	\end{multicols}
	
	\vspace*{-\baselineskip}
	
	\hangpara{\small (Assume same nutrition, etc. for both species.)}
	
	\vskip0pt plus 1filll
	
	\tiny \url{http://oceanexplorer.noaa.gov/explorations/02alaska/logs/jun27/jun27.html}
\end{frame}

\begin{frame}[t]{Which species produces more offspring and why?}
	\begin{multicols}{2}

	\begin{center}
		\includegraphics[height=0.45\textheight]{larval_strategies_lecitho_eggs}
		
		\textit{Lithodes couesi}\\
		lecithotrophic

	\end{center}	
	\columnbreak

	\begin{center}	
		\includegraphics[height=0.45\textheight]{larval_strategies_plankto_eggs}

		\textit{Macroregonia macrochira}\\
		planktotrophic
	\end{center}
	\end{multicols}
	
%	\hangpara{\small (Assume nutrition, etc. is same for both species.)}
	
	\vskip0pt plus 1filll
	
	\tiny \url{http://oceanexplorer.noaa.gov/explorations/02alaska/logs/jun27/jun27.html}
\end{frame}

\begin{frame}[t]{Larval type often correlates with species range size.}

\includegraphics[width=\textwidth]{larval_dispersal}

\vskip0pt plus 1filll

\tiny \textcopyright\,Levinton, Oxford Univ. Press.
\end{frame}

\begin{frame}[t]{Why do planktotrophic larvae become less common at high latitude?}

\includegraphics[width=\textwidth]{larval_strategy_latitude}

\vskip0pt plus 1filll

\tiny \textcopyright\,Levinton, Oxford Univ. Press.
\end{frame}


\end{document}
