%!TEX TS-program = lualatex
%!TEX encoding = UTF-8 Unicode

\documentclass[t]{beamer}

%%%% HANDOUTS For online Uncomment the following four lines for handout
%\documentclass[t,handout]{beamer}  %Use this for handouts.
%\usepackage{handoutWithNotes}
%\pgfpagesuselayout{3 on 1 with notes}[letterpaper,border shrink=5mm]
%	\setbeamercolor{background canvas}{bg=black!5}


%%% Including only some slides for students.
%%% Uncomment the following line. For the slides,
%%% use the labels shown below the command.
%\includeonlylecture{student}

%% For students, use \lecture{student}{student}
%% For mine, use \lecture{instructor}{instructor}


%\usepackage{pgf,pgfpages}
%\pgfpagesuselayout{4 on 1}[letterpaper,border shrink=5mm]

% FONTS
\usepackage{fontspec}
\def\mainfont{Linux Biolinum O}
\setmainfont[Ligatures={Common,TeX}, BoldFont={* Bold}, ItalicFont={* Italic}, Numbers={Proportional}]{\mainfont}
\setsansfont[Scale=MatchLowercase]{Linux Biolinum O} 
\usepackage{microtype}

\usepackage{graphicx}
	\graphicspath{%
	{/Users/mtaylor/Pictures/teach/348/lectures/}%
	{/Users/mtaylor/Pictures/teach/common/}%}%
	{img/}} % set of paths to search for images

\usepackage{amsmath,amssymb}

%\usepackage{units}

%\usepackage{booktabs}
\usepackage{multicol}
%	\setlength{\columnsep=1em}

%\usepackage{tikz}
%	\tikzstyle{every picture}+=[remember picture,overlay]

%\usepackage[version=4]{mhchem}
%\usepackage{siunitx}

%\usepackage{textcomp}
%\usepackage{setspace}
\mode<presentation>
{
  \usetheme{Lecture}
  \setbeamercovered{invisible}
  \setbeamertemplate{items}[square]
}

\usepackage{hyperref}

\newcommand\HiddenWord[1]{%
	\alt<handout>{\rule{4cm}{0.4pt}}{#1}%
}



\begin{document}

\lecture{student}{student}

{\usebackgroundtemplate{\includegraphics[width=\paperwidth]{tide_intro}}
\begin{frame}[b]{Tide goes in, tide goes out\dots You can't explain that.}

\tiny\textcolor{white}{John Arnold, Flickr Creative Commons.\hfill\href{https://www.youtube.com/watch?v=wb3AFMe2OQY&t=110s}{Link to Video: 1:50}}
\end{frame}
}

\begin{frame}[t]{Tides are the regular movement of sea water due to gravity.}

\vspace*{-\baselineskip}

\begin{multicols}{2}

	\begin{center}
		\includegraphics[width=0.48\textwidth]{tide_mechanics}
	\end{center}

\columnbreak

	\hangpara Moon and earth rotate as one around a common point.

	\hangpara Earth also rotates around it’s own axis.

	\hangpara Wobble causes bulge opposite of moon.

	\hangpara Moon gravity causes opposing bulge.
\end{multicols}

	\vskip0pt plus 1filll

	\tiny\textcopyright\,McGraw-Hill

\end{frame}


\begin{frame}[t]{The tidal cycle lasts just longer than 24 hours.}
	\vspace*{-\baselineskip}

	\begin{center}
		\includegraphics[height=0.85\textheight]{tide_cycle} 
	\end{center}

	\vskip0pt plus 1filll

	\tiny\textcopyright\,McGraw-Hill

\end{frame}

\begin{frame}[t]{One of three tidal patterns are possible.}

	\vspace*{-\baselineskip}

	\begin{center}
		\includegraphics[width=\textwidth]{tidal_patterns} 
	\end{center}

	\vskip0pt plus 1filll

	\tiny\textcopyright\,McGraw-Hill

\end{frame}

\begin{frame}{Diurnal tides have one high and one low tide each lunar day.}
\includegraphics[width=\linewidth]{tide_diurnal_adak}

\hangpara One lunar day is 24 hours and 50 minutes.
\end{frame}

\begin{frame}{Semidiurnal tides have two high and two low tides of \emph{roughly} equal heights each lunar day.}
\includegraphics[width=\linewidth]{tide_semidiurnal_bridgeport}

\hangpara Similar height difference between sequential high and low tides.
\end{frame}

\begin{frame}{Mixed semidiurnal tides have two high and two low tides of very different heights each lunar day.}
\includegraphics[width=\linewidth]{tide_mixed_semi_garibaldi}

\hangpara Distinct height difference between sequential high and low tides.
\end{frame}

{\usebackgroundtemplate{\includegraphics[width=\paperwidth]{tidal_patterns_distribution}}
\begin{frame}[b]{Different patterns occur in different regions.}
\hfill\tiny\textcopyright\,McGraw-Hill
\end{frame}
}

\begin{frame}[t]{Tides vary weekly between \highlight{spring} and \highlight{neap} tides.}

	\vspace*{-\baselineskip}

	\begin{center}
		\includegraphics[width=\textwidth]{tide_spring_neap} 
	\end{center}

\vskip0pt plus 1filll

\tiny\textcopyright\,McGraw-Hill

\end{frame}


{\usebackgroundtemplate{\includegraphics[width=\paperwidth]{rocky_intertidal_intro}}
\begin{frame}[b]{Rocky Intertidal Communities}
\tiny\textcolor{white}{Source unknown.}
\end{frame}
}


\begin{frame}[t]{The rocky intertidal has three zones.}

\vspace*{-\baselineskip}

\begin{multicols}{2}

	\begin{center}
		\includegraphics[width=0.48\textwidth]{rocky_intertidal_zones}
	\end{center}

\columnbreak

	\hangpara\highlight{Supralittoral fringe}\\
	\hspace*{1em}splash zone\\
	\hspace*{1em}periwinkles\\
	\hspace*{1em}encrusting algae and lichens.
	

	\hangpara\highlight{Midlittoral zone}\\
	\hspace*{1em}greatest diversity overall\\
	\hspace*{1em}barnacles and mussels dominate.

	\hangpara\highlight{Infralittoral zone}\\
	\hspace*{1em}aka \highlight{sublittoral fringe}\\
	\hspace*{1em}brown macroalgae dominate.

\end{multicols}

	\vskip0pt plus 1filll

	\tiny\textcopyright\,McGraw-Hill

\end{frame}

{\usebackgroundtemplate{\includegraphics[width=\paperwidth]{rocky_zones_identified}}
\begin{frame}[b]{}
\hfill\tiny Bcasterline, Wikimedia, Public Domain.
\end{frame}
}


{\usebackgroundtemplate{\includegraphics[width=\paperwidth]{rocky_intertidal_zonation}}
\begin{frame}[b]{What biotic and abiotic factors cause zonation?}
\tiny\textcolor{white}{Ken Clifton, Flickr Creative Commons.}
\end{frame}
}


\begin{frame}[t]{The primary causes of zonation are}

	\begin{multicols}{2}

		\onslide<1->\hangpara\textbf{Abiotic}
		
		\onslide<2->\hangpara\HiddenWord{Dessication}
	
		\onslide<3->\hangpara\HiddenWord{Temperature (\& light)}
	
		\onslide<4->\hangpara\HiddenWord{Wave action}
		
		\onslide<4->\hangpara\HiddenWord{}

	\columnbreak
	
		\onslide<1->\hangpara\textbf{Biotic}
		
		\onslide<5->\hangpara\HiddenWord{Competition}
		
		\onslide<6->\hangpara\HiddenWord{Predation, inc. grazing}
		
		\onslide<6->\hangpara\HiddenWord{}
		
		\onslide<6->\hangpara\HiddenWord{}

	\end{multicols}
	
\end{frame}

\begin{frame}[t]{Abiotic and biotic factors interact to create community structure.}

	\vspace*{-\baselineskip}

	\begin{center}
		\includegraphics[width=\textwidth]{rocky_barnacles_example} 
	\end{center}

\vskip0pt plus 1filll

\tiny\textcopyright\,McGraw-Hill

\end{frame}


\begin{frame}[t]{\highlight{Keystone species} increase diversity in the rocky intertidal.}

	\vspace*{-\baselineskip}

	\begin{center}
		\includegraphics[width=\textwidth]{rocky_keystone_species} 
	\end{center}

	\vspace*{-0.5\baselineskip}
	
	\textit{Pisaster} seastars are a rocky intertidal keystone species.
	
\vskip0pt plus 1filll

\tiny\textcopyright\,McGraw-Hill

\end{frame}


{\usebackgroundtemplate{\includegraphics[width=\paperwidth]{rocky_pisaster}}
\begin{frame}[b]{}
\tiny\hfill\textcolor{white}{Source unknown.}
\end{frame}
}

\begin{frame}[t]{\highlight{Patch dynamics} is influenced by the size of the patch.}

	\vspace*{-\baselineskip}

	\begin{center}
		\includegraphics[width=\textwidth]{rocky_patch_dynamics} 
	\end{center}


\vskip0pt plus 1filll

\hfill\tiny\textcopyright\,Oxford

\end{frame}


\begin{frame}[t]{Communities can switch between \highlight{alternate stable states.}}

	\vspace*{-\baselineskip}

	\begin{center}
		\includegraphics[width=\textwidth]{rocky_alternate_stable_states} 
	\end{center}


\vskip0pt plus 1filll

\hfill\tiny\textcopyright\,Oxford

\end{frame}



\begin{frame}[t]{Compare the rocky intertidal and subtidal zones.}

\vspace*{-\baselineskip}

\begin{multicols}{2}

	\vspace*{-\baselineskip}
	\begin{center}
		\includegraphics[width=0.45\textwidth]{rocky_compare}
	\end{center}

\columnbreak

	\hangpara Discuss the abiotic and biotic factors that influence dynamics at the upper limit and at the lower limit of each community. 
	

	\hangpara Do you observe any trends or commonalities in the types of factors that control the upper and lower limits of these communities? Explain.

\end{multicols}

	\vskip0pt plus 1filll

	\tiny\textcopyright\,McGraw-Hill

\end{frame}


\end{document}
