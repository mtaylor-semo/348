%!TEX TS-program = lualatex
%!TEX encoding = UTF-8 Unicode

\documentclass[t]{beamer}

%%%% HANDOUTS For online Uncomment the following four lines for handout
%\documentclass[t,handout]{beamer}  %Use this for handouts.
%\includeonlylecture{student}
%\usepackage{handoutWithNotes}
%\pgfpagesuselayout{3 on 1 with notes}[letterpaper,border shrink=5mm]
%	\setbeamercolor{background canvas}{bg=black!5}


%%% Including only some slides for students.
%%% Uncomment the following line. For the slides,
%%% use the labels shown below the command.
%\includeonlylecture{student}

%% For students, use \lecture{student}{student}
%% For mine, use \lecture{instructor}{instructor}


%\usepackage{pgf,pgfpages}
%\pgfpagesuselayout{4 on 1}[letterpaper,border shrink=5mm]

% FONTS
\usepackage{fontspec}
\def\mainfont{Linux Biolinum O}
\setmainfont[Ligatures={Common,TeX}, Contextuals={NoAlternate}, BoldFont={* Bold}, ItalicFont={* Italic}, Numbers={Proportional}]{\mainfont}
\setsansfont[Scale=MatchLowercase]{Linux Biolinum O} 
\usepackage{microtype}

\usepackage{graphicx}
	\graphicspath{%
	{/Users/goby/Pictures/teach/348/lectures/}%
	{/Users/goby/Pictures/teach/common/}%}%
	{img/}} % set of paths to search for images

\usepackage{amsmath,amssymb}

%\usepackage{units}

\usepackage{booktabs}
\usepackage{multicol}
%	\setlength{\columnsep=1em}

%\usepackage{tikz}
%	\tikzstyle{every picture}+=[remember picture,overlay]

%\usepackage{chemfig}
%\usepackage[version=4]{mhchem}

%\usepackage{textcomp}
%\usepackage{setspace}
\mode<presentation>
{
  \usetheme{Lecture}
  \setbeamercovered{invisible}
  \setbeamertemplate{items}[square]
}

\usepackage{animate}

\usepackage{calc} % Needed for \widthof
\usepackage{hyperref}

\newcommand\HiddenWord[1]{%
	\alt<handout>{\rule{\widthof{#1}}{\fboxrule}}{#1}%
}



\begin{document}

\lecture{student}{student}

{\usebackgroundtemplate{\includegraphics[width=\paperwidth]{deep_sea_intro}}
\begin{frame}[t]{\textcolor{white}{The \textcolor{orange5}{deep sea} is the largest habitat on Earth.}}
\end{frame}}

{\usebackgroundtemplate{\includegraphics[width=\paperwidth]{deep_sea_parameters}}
\begin{frame}[t]{\textcolor{white}{List the environmental parameters that affect the biology of deep-sea organisms.}}

\hangpara\parbox{0.45\textwidth}{\raggedright\textcolor{white}{Then, pick one that you think most affects the deep-sea community interactions and explain why.}}

\hangpara\parbox{0.45\textwidth}{\textcolor{white}{Do this for both the mesopelagic and the bathypelagic communities.}}

\end{frame}}

{\usebackgroundtemplate{\includegraphics[width=\paperwidth]{thermocline_halocline}}
\begin{frame}[b]{Temperature is uniform below the thermocline.}
\tiny\textcopyright\, McGraw-Hill
\end{frame}}

{\usebackgroundtemplate{\includegraphics[width=\paperwidth]{temperature_sst_gradient}}
\begin{frame}[b]{If thermocline prevents mixing, how does O\textsubscript{2} get to deep sea?}
\tiny NOAA, Public Domain.
\end{frame}}

{\usebackgroundtemplate{\includegraphics[width=\paperwidth]{sea_surface_density}}
\begin{frame}[b]{Surface density is greatest in North Atlantic and parts of the Southern Ocean.}
\tiny NASA, Public Domain.
\end{frame}}


{\usebackgroundtemplate{\includegraphics[width=\paperwidth]{thermohaline_circulation}}
\begin{frame}[b]
\tiny Avsa, Wikimedia Commons.
\end{frame}}


{\usebackgroundtemplate{\includegraphics[width=\paperwidth]{depth_pressure}}
\begin{frame}[b]
\tiny\textcopyright\,McGraw-Hill
\end{frame}}

{\usebackgroundtemplate{\includegraphics[width=\paperwidth]{depth_light}}
\begin{frame}[b]
\tiny\textcopyright\,McGraw-Hill
\end{frame}}

{\usebackgroundtemplate{\includegraphics[width=\paperwidth]{depth_light_intensity}}
\begin{frame}[b]{Light intensity decreases with depth and water clarity.}
\tiny\textcopyright\,Benjamin-Cummings
\end{frame}}

{\usebackgroundtemplate{\includegraphics[width=\paperwidth]{deep_sea_adaptations_color_shape}}
\begin{frame}[t]{Several color types and body shape are found in deep sea organisms.}

\vspace*{-\baselineskip}

\hangpara Transparent zooplankton.

\hangpara Black or red non-migrators.

\hangpara\parbox{0.45\textwidth}{Countershaded and compressed migrators.}

\hangpara Jelly-like aphotic species.

\end{frame}}


{\usebackgroundtemplate{\includegraphics[width=\paperwidth]{deep_sea_adaptations_vision}}
\begin{frame}[t]{Vision adapted to capture limited light.}

	\hangpara\parbox{0.48\textwidth}{Large eyes are 15–30\times\ more sensitive than human eyes.}

	\hangpara\parbox{0.48\textwidth}{Tubular eyes may have dual lenses and retinas.}

\end{frame}}

{\usebackgroundtemplate{\includegraphics[width=\paperwidth]{deep_sea_adaptations_biolum}}
\begin{frame}[t]{\textcolor{white}{Bioluminescence is chemical light produced by living organisms.}}

\end{frame}}

\lecture{instructor}{instructor}
{\setbeamercolor{background canvas}{bg=black}
\begin{frame}[c]

	\begin{center}
		\animategraphics[autoplay, loop,width=0.6\textwidth]{4}{copepod_flash_}{0}{30}
	\end{center}
	
\end{frame}}


{\setbeamercolor{background canvas}{bg=black}
\begin{frame}[c]
	\begin{multicols}{2}
		\animategraphics[autoplay, loop,width=0.5\textwidth]{4}{flashlight_fish_}{0}{24}
	\columnbreak
		\includegraphics[width=0.49\textwidth]{flashlight_fish}
	\end{multicols}

\vfilll
\hfill \tiny\textcolor{white}{\href{https://youtu.be/VlyGpg35jMA}{Link to video}}
\end{frame}}



\lecture{student}{student}

{\usebackgroundtemplate{\includegraphics[width=\paperwidth]{epipelagic_nutrient_sources}}
\begin{frame}[t]{About \highlight{20\%} of epipelagic NPP reaches the deep sea.}

\vspace*{63mm}

\hangpara Low surface NPP equals low deep-sea energy and biomass.

\vskip0pt plus 1filll

\hfill\tiny\textcopyright\,Benjamin-Cummings.
\end{frame}}


{\usebackgroundtemplate{\includegraphics[width=\paperwidth]{deep_sea_adaptations_teeth}}
\begin{frame}[t]{Feeding adapted to increase capture success of scarce resources.}

\hangpara\parbox{0.45\textwidth}{Many species have large, sharp teeth.}

\hangpara\parbox{0.45\textwidth}{Many species have large mouths with hinged jaws.}

\vskip0pt plus 1filll

\tiny Upper: \textcopyright\,Norfanz. Lower: Source unknown.
\end{frame}}


{\usebackgroundtemplate{\includegraphics[width=\paperwidth]{deep_sea_adaptations_stomach}}
\begin{frame}[b]{\textcolor{white}{Many aphotic fishes have enlarged stomachs.}}

\tiny\textcolor{white}{\textcopyright\,Natural History Museum, London.}
\end{frame}}

{\usebackgroundtemplate{\includegraphics[width=\paperwidth]{deep_sea_cool_examples}}
\begin{frame}[b]
\end{frame}}


\end{document}
