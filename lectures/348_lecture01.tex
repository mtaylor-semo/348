%!TEX TS-program = lualatex
%!TEX encoding = UTF-8 Unicode

\documentclass[t]{beamer}

%%%% HANDOUTS For online Uncomment the following four lines for handout
%\documentclass[t,handout]{beamer}  %Use this for handouts.
%\usepackage{handoutWithNotes}
%\pgfpagesuselayout{3 on 1 with notes}[letterpaper,border shrink=5mm]


%%% Including only some slides for students.
%%% Uncomment the following line. For the slides,
%%% use the labels shown below the command.
%\includeonlylecture{student}

%% For students, use \lecture{student}{student}
%% For mine, use \lecture{instructor}{instructor}


%\usepackage{pgf,pgfpages}
%\pgfpagesuselayout{4 on 1}[letterpaper,border shrink=5mm]

% FONTS
\usepackage{fontspec}
\def\mainfont{Linux Biolinum O}
\setmainfont[Ligatures={Common,TeX}, Contextuals={NoAlternate}, BoldFont={* Bold}, ItalicFont={* Italic}, Numbers={Proportional}]{\mainfont}
\setsansfont[Scale=MatchLowercase]{Linux Biolinum O} 
\usepackage{microtype}

\usepackage{graphicx}
	\graphicspath{%
	{/Users/goby/Pictures/teach/348/lectures/}%
	{/Users/goby/Pictures/teach/common/}%}%
	{img/}} % set of paths to search for images

\usepackage{amsmath,amssymb}

%\usepackage{units}

\usepackage{booktabs}
\usepackage{multicol}
%	\setlength{\columnsep=1em}

%\usepackage{textcomp}
%\usepackage{setspace}
\usepackage{tikz}
	\tikzstyle{every picture}+=[remember picture,overlay]

\mode<presentation>
{
  \usetheme{Lecture}
  \setbeamercovered{invisible}
  \setbeamertemplate{items}[square]
}

%\usepackage{calc}
\usepackage{hyperref}

\newcommand\HiddenWord[1]{%
	\alt<handout>{\rule{\widthof{#1}}{\fboxrule}}{#1}%
}



\begin{document}

%\lecture{instructor}{instructor}
\lecture{student}{student}

{
\usebackgroundtemplate{\includegraphics[width=\paperwidth]{marine_biology_intro}}
\begin{frame}[t]
\end{frame}
}

{
\usebackgroundtemplate{\includegraphics[width=\paperwidth]{mike_snake}}
\begin{frame}[t,plain]
\large
\vspace{5ex}
\hangpara\hspace{17em} Mike Taylor

\hangpara\hspace{17em} RH 217

\hangpara\hspace{17em} mtaylor@semo.edu

\end{frame}
}

\begin{frame}[t]{You \highlight{earn} your grade with}
	\begin{center}\large\begin{tabular}{@{}ll@{}}
	Four 75-point exams & 60\% \\
	Home/in-class Assignments & 20\% \\
	5-in-5 Presentation & 20\% \\
	\end{tabular}
	\end{center}
\end{frame}

{
\usebackgroundtemplate{\includegraphics[width=\paperwidth]{5in5_overview}
}
\begin{frame}[t]
\end{frame}
}

\begin{frame}[t]{Your \highlight{objectives} for this course are to}
	\hangpara learn basic terms and concepts,

	\hangpara learn basic oceanographic processes,
	
	\hangpara relate oceanographic processes to biological processes, and

	\hangpara develop a biological context for marine conservation.
	
	\pause
	
	\hangpara \highlight{I hope!}

\end{frame}

{\usebackgroundtemplate{\includegraphics[width=\paperwidth]{cocos_jacks}}
\begin{frame}[t]{What is the \highlight{marine} environment?}
\end{frame}
}

{\usebackgroundtemplate{\includegraphics[width=\paperwidth]{mussel_research}}
\begin{frame}[t]
\end{frame}
}

{\usebackgroundtemplate{\includegraphics[width=\paperwidth]{beach_seining}}
\begin{frame}[t]
\end{frame}
}

{\usebackgroundtemplate{\includegraphics[width=\paperwidth]{seagrass_transect}}
\begin{frame}[t]
\end{frame}
}

{\usebackgroundtemplate{\includegraphics[width=\paperwidth]{reef_survey}}
\begin{frame}[t]
\end{frame}
}

{\usebackgroundtemplate{\includegraphics[width=\paperwidth]{noaa_aquarius}}
\begin{frame}[t]
\end{frame}
}

{\usebackgroundtemplate{\includegraphics[width=\paperwidth]{rv_pelican}}
\begin{frame}[t]
\end{frame}
}

{\usebackgroundtemplate{\includegraphics[width=\paperwidth]{rv_seward_johnson}}
\begin{frame}[t]
\end{frame}
}

{\usebackgroundtemplate{\includegraphics[width=\paperwidth]{rv_flip_horizontal}}
\begin{frame}[t]
\end{frame}
}

{\usebackgroundtemplate{\includegraphics[width=\paperwidth]{rv_flip_upright}}
\begin{frame}[b]
\tiny\href{http://www.youtube.com/watch?v=tQxQfQU_hsk}{Link to video}
\end{frame}
}

{\usebackgroundtemplate{\includegraphics[width=\paperwidth]{rov_hercules}}
\begin{frame}[t]
\end{frame}
}

{\usebackgroundtemplate{\includegraphics[width=\paperwidth]{rv_sealink}}
\begin{frame}[t]
\end{frame}
}


{\usebackgroundtemplate{\includegraphics[width=\paperwidth]{dsv_alvin}}
\begin{frame}[t]
\end{frame}
}

\lecture{instructor}{instructor}

{\usebackgroundtemplate{\includegraphics[width=\paperwidth]{vent_community}}
\begin{frame}[b]{\textcolor{white}{The Alvin discovered hydrothermal vent communities.}}

\hfill \tiny \textcolor{white}{Ocean Networks Canada, Flickr, \ccbyncsa{2}}
\end{frame}
}


\lecture{student}{student}

{\usebackgroundtemplate{\includegraphics[width=\paperwidth]{bathyscaphe_trieste}}
\begin{frame}[t]
\end{frame}
}

{\usebackgroundtemplate{\includegraphics[width=\paperwidth]{deepsea_challenger_above}}
\begin{frame}[t]
\end{frame}
}

{\usebackgroundtemplate{\includegraphics[width=\paperwidth]{deepsea_challenger_landed}}
\begin{frame}[t]
\end{frame}
}


\lecture{instructor}{instructor}

\begin{frame}[t]{Your \highlight{homework} assignment is}
	\hangpara Name all of the recognized oceans, and

	\hangpara identify the deepest spot in each ocean, with depth (in meters).
\end{frame}

\end{document}
