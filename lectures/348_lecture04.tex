%!TEX TS-program = lualatex
%!TEX encoding = UTF-8 Unicode

%\documentclass[t]{beamer}

%%%% HANDOUTS For online Uncomment the following four lines for handout
\documentclass[t,handout]{beamer}  %Use this for handouts.
\usepackage{handoutWithNotes}
\pgfpagesuselayout{3 on 1 with notes}[letterpaper,border shrink=5mm]
%	\setbeamercolor{background canvas}{bg=black!5}


%%% Including only some slides for students.
%%% Uncomment the following line. For the slides,
%%% use the labels shown below the command.
%\includeonlylecture{student}

%% For students, use \lecture{student}{student}
%% For mine, use \lecture{instructor}{instructor}


%\usepackage{pgf,pgfpages}
%\pgfpagesuselayout{4 on 1}[letterpaper,border shrink=5mm]

% FONTS
\usepackage{fontspec}
\def\mainfont{Linux Biolinum O}
\setmainfont[Ligatures={Common,TeX}, Contextuals={NoAlternate}, BoldFont={* Bold}, ItalicFont={* Italic}, Numbers={Proportional}]{\mainfont}
\setsansfont[Scale=MatchLowercase]{Linux Biolinum O} 
\usepackage{microtype}

\usepackage{graphicx}
	\graphicspath{%
	{/Users/goby/Pictures/teach/348/lectures/}%
	{/Users/goby/Pictures/teach/common/}%}%
	{img/}} % set of paths to search for images

\usepackage{amsmath,amssymb}

%\usepackage{units}

\usepackage{booktabs}
\usepackage{multicol}
%	\setlength{\columnsep=1em}

%\usepackage{textcomp}
%\usepackage{setspace}
\usepackage{tikz}
	\tikzstyle{every picture}+=[remember picture,overlay]

\mode<presentation>
{
  \usetheme{Lecture}
  \setbeamercovered{invisible}
  \setbeamertemplate{items}[square]
}

\usepackage{calc} % Needed for \widthof
\usepackage{hyperref}

\newcommand\HiddenWord[1]{%
	\alt<handout>{\rule{\widthof{#1}}{\fboxrule}}{#1}%
}



\begin{document}

%\lecture{student}{student}

{\usebackgroundtemplate{\includegraphics[width=\paperwidth]{plankton_vs_nekton}}
\begin{frame}[b]
\tiny\textcolor{white}{Betty White, Wikimedia Commons\hfill Public domain, NOAA Photo Library}
\end{frame}
}


\begin{frame}[t]{Nekton swim independent of ocean current.}
	\vspace*{-\baselineskip}
	
	\begin{multicols}{2}
		\hangpara\highlight{Holoepipelagic}
		
		\hangpara\highlight{Meroepipelagic}

		\vfill
		
		\includegraphics[width=0.44\textwidth]{nekton_whale}
	
	\columnbreak
	
		\includegraphics[width=0.44\textwidth]{nekton_tuna}

		\reflectbox{\includegraphics[width=0.44\textwidth]{nekton_nautilus}}
		
	\end{multicols}

\tiny \hfill Tuna \textcopyright\,Pearson

\tiny Whale \textcopyright\,Pearson Education\hfill Nautilus, Hans Hillewaert, Wikimedia Commons.
\end{frame}

\begin{frame}[t]{Plankton drift with the ocean current.}
	\vspace*{-\baselineskip}
	
	\begin{multicols}{2}
		\hangpara Phytoplankton
		
		\hangpara Zooplankton

		\hangpara Mixoplankton

		\vspace*{\baselineskip}
		
		\hangpara Net plankton

		\hangpara Nanoplankton
	
	\columnbreak
	
		\hangpara\highlight{Holoplankton}

		\hangpara\highlight{Meroplankton}
		
		\vspace*{\baselineskip}
		
		\includegraphics[width=0.40\textwidth]{plankton_net}

	\end{multicols}

\vskip0pt plus 1filll

\hfill\tiny NOAA, Flickr Creative Commons.
\end{frame}



\begin{frame}[t]{Given the parameters at right:}
	\begin{multicols}{2}
		\hangpara Describe 2–3 nekton adaptations that increase success as predator or prey in this environment

		\hangpara Do the same for plankton.
	\columnbreak
	
		Lots of light\\
		Clear water\\
		No structure\\
		Vast area\\
		Water density\\
		Low diversity
	\end{multicols}

\end{frame}

\begin{frame}[t]{Fast swimming predators are energy efficient in dense medium.}

	\includegraphics[width=\textwidth]{swimming_tuna}

	\vskip0pt plus 1filll

	\tiny Niall Kennedy, Flickr Creative Commons.
\end{frame}

\begin{frame}[t]{Slow swimming filter feeders are less efficient.}

	\includegraphics[width=\textwidth]{feeding_anchovy}

	\vskip0pt plus 1filll

	\tiny \textcopyright,Monterey Bay Aquarium.
\end{frame}


{\usebackgroundtemplate{\includegraphics[width=\paperwidth]{silvery_flying_fish}}
\begin{frame}[b]
\tiny Dan Irizarry, Flickr Creative Commons
\end{frame}
}

{\usebackgroundtemplate{\includegraphics[width=\paperwidth]{transparent_pelagic_gastropod}}
\begin{frame}[b]
\tiny\textcolor{white}{Johnson Lab, Duke University, \url{http://sites.biology.duke.edu/johnsenlab/}}
\end{frame}
}

\begin{frame}[t]{Nekton maintain buoyancy through different means.}
	\begin{multicols}{2}

		\includegraphics[width=0.49\textwidth]{buoyancy_bladders}

		\vspace*{2\baselineskip}
		
		\includegraphics[width=0.49\textwidth]{buoyancy_shark}

	\columnbreak
	
		\hangpara Swim bladders, lungs, air sacs
		
		\hangpara Fins and body shape
		
		\hangpara Lipids
		
		\hangpara Ions

	\end{multicols}

\end{frame}

\begin{frame}[t]{Plankton maintain buoyancy through different means.}
	\begin{multicols}{2}

		\includegraphics[width=0.49\textwidth]{janthina}

%		\vspace*{2\baselineskip}
		
%		\includegraphics[height=0.4\textheight]{Glaucus_atlanticus}

	\columnbreak
	
		\hangpara Floatation devices
		
		\hangpara Surface area
		
		\hangpara Lipids
		
		\hangpara Ions

	\end{multicols}

\end{frame}

{\usebackgroundtemplate{\includegraphics[width=\paperwidth]{Glaucus_atlanticus}}
\begin{frame}[b]
\hfill\tiny\textcolor{white}{Sylke Rohrlach, Wikimedia Commons}
\end{frame}
}

{\usebackgroundtemplate{\includegraphics[width=\paperwidth]{foraminifera}}
\begin{frame}[b]
\tiny\textcolor{white}{\textcopyright\,McGraw-Hill}
\end{frame}
}



\end{document}
