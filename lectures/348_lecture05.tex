%!TEX TS-program = lualatex
%!TEX encoding = UTF-8 Unicode

\documentclass[t]{beamer}

%%%% HANDOUTS For online Uncomment the following four lines for handout
%\documentclass[t,handout]{beamer}  %Use this for handouts.
%\usepackage{handoutWithNotes}
%\pgfpagesuselayout{3 on 1 with notes}[letterpaper,border shrink=5mm]
%	\setbeamercolor{background canvas}{bg=black!5}


%%% Including only some slides for students.
%%% Uncomment the following line. For the slides,
%%% use the labels shown below the command.
\includeonlylecture{student}

%% For students, use \lecture{student}{student}
%% For mine, use \lecture{instructor}{instructor}


%\usepackage{pgf,pgfpages}
%\pgfpagesuselayout{4 on 1}[letterpaper,border shrink=5mm]

% FONTS
\usepackage{fontspec}
\def\mainfont{Linux Biolinum O}
\setmainfont[Ligatures={Common,TeX}, Contextuals={NoAlternate}, BoldFont={* Bold}, ItalicFont={* Italic}, Numbers={Proportional}]{\mainfont}
\setsansfont[Scale=MatchLowercase]{Linux Biolinum O} 
\usepackage{microtype}

\usepackage{graphicx}
	\graphicspath{%
	{/Users/goby/Pictures/teach/348/lectures/}%
	{/Users/goby/Pictures/teach/common/}%}%
	{img/}} % set of paths to search for images

\usepackage{amsmath,amssymb}

%\usepackage{units}

\usepackage{booktabs}
\usepackage{multicol}
%	\setlength{\columnsep=1em}

\usepackage{tikz}
	\tikzstyle{every picture}+=[remember picture,overlay]

\usepackage{chemfig}
\usepackage[version=4]{mhchem}

%\usepackage{textcomp}
%\usepackage{setspace}
\mode<presentation>
{
  \usetheme{Lecture}
  \setbeamercovered{invisible}
  \setbeamertemplate{items}[square]
}

\usepackage{calc} % Needed for \widthof
\usepackage{hyperref}

\newcommand\HiddenWord[1]{%
	\alt<handout>{\rule{\widthof{#1}}{\fboxrule}}{#1}%
}



\begin{document}

\lecture{student}{student}

\begin{frame}[t]{Two major classes of phytoplankton are }
	\vspace*{-\baselineskip}
	
	\begin{multicols}{2}
		{\centering
		\includegraphics[width=0.36\textwidth]{diatoms}
		
		\vspace*{0.5\baselineskip}
		
		\includegraphics[width=0.46\textwidth]{dinoflagellates}}
		
	\columnbreak
	
		\hangpara\highlight{Diatoms:} Bacillariophyceae

		\vspace*{\baselineskip}

		\hangpara\highlight{Dinoflagellates:} Dinophyceae
		
	\end{multicols}

\vskip0pt plus 1filll

\tiny top: USGS Photos from Life; bottom: \textcopyright\,McGraw-Hill.
\end{frame}


\begin{frame}[t]{Define primary productivity.}

\begin{center}
	\ce{CO2 + 6H2O ->[energy] C6H12O6 + 6O2}
	
	\vspace*{0.5\baselineskip}
	
	\includegraphics[width=\textwidth]{global_primary_productivity}
\end{center}

\end{frame}

\begin{frame}[t]{Terrestrial and oceanic ecosystems have similar NPP. Why?}

	\vspace*{0.5\baselineskip}

	\includegraphics[width=\textwidth]{global_primary_productivity}

\end{frame}

\begin{frame}[t]{What limits primary productivity in the ocean?}

\begin{center}
%	\ce{CO2 + 6H2O ->[energy] C6H12O6 + 6O2}
	
	\vspace*{0.5\baselineskip}
	
	\includegraphics[width=\textwidth]{global_primary_productivity}
\end{center}

\end{frame}


\begin{frame}[t]{\highlight{Photosynthetically active radiation} affects GPP.}
\vspace*{-\baselineskip}
\begin{center}
	\includegraphics[height=0.85\textheight]{light_penetration}
\end{center}

\vskip0pt plus 1filll

\tiny\textcopyright\,Benjamin-Cummings
\end{frame}



\begin{frame}[t]{In pairs, plot GPP vs depth.}
\end{frame}

\lecture{instructor}{instructor}
\begin{frame}[t]{Why is GPP maximum around 20m depth?}
	\vspace*{-\baselineskip}
	\begin{center}
		\includegraphics[height=0.85\textheight]{gpp_vs_depth}
	\end{center}

	\vskip0pt plus 1filll

	\tiny\textcopyright\,Benjamin-Cummings
\end{frame}

\lecture{student}{student}

\begin{frame}[t]{Marine organisms require three major nutrients.}

	\begin{multicols}{2}

		\begin{center}
			\chemfig{C(=[:180]O)(=[:0]O)}
			
			\vspace*{4\baselineskip}
			
			\chemfig{C(=[:90]O)(-[:210]OH)(-[:330]OH)}
		\end{center}
				
	\columnbreak
	
		\hangpara Carbon (C)\\ Nitrogen (N)\\ Phosphorus (P)
		
		\hangpara Dissolved or Particulate
		
		\hangpara Organic or Inorganic
		
	\end{multicols}
\end{frame}

\begin{frame}[t]{The \highlight{Redfield ratio} shows the ratio of C:N:P atoms in phytoplankton.}

\begin{center}
	\begin{tabular}{@{}lrr@{}}
	\toprule
	Atom & Phytoplankton & Water \\
	\midrule
	Carbon & 106 & 106 \\
	Nitrogen & 16 & 14.7 \\
	Phosphorus & 1 & 1 \\
	\bottomrule
	\end{tabular}
\end{center}

\hangpara Which nutrient is limiting?

\end{frame}


\begin{frame}[t]{Nutrients are not uniformly distributed with depth.}

	\includegraphics[width=\textwidth]{nutrient_depth_distribution}

\vskip0pt plus 1filll

\tiny\textcopyright\,Marine Ecological Processes, Springer.
\end{frame}

\begin{frame}[t]{Vertical mixing returns nutrients to surface.}

	\includegraphics[width=\textwidth]{vertical_nutrient_mixing}

\vskip0pt plus 1filll

\tiny\textcopyright\,Benjamin-Cummings.
\end{frame}

{\usebackgroundtemplate{\includegraphics[width=\paperwidth]{spring_algal_blooms}}
\begin{frame}[b]{Mixing increases primary productivity in the spring.}
\tiny\hfill\textcopyright\,Oxford Univ. Press.
\end{frame}
}

{\usebackgroundtemplate{\includegraphics[width=\paperwidth]{copepod_calanus}}
\begin{frame}[b]{\textcolor{white}{Copepods are 70–90\% of total ocean biomass.}}
\tiny\textcolor{white}{Uwe Kils, Wikimedia Commons}
\end{frame}
}

\begin{frame}[t]{One copepod can consume \textasciitilde12,000 diatom cells per hour.}
	\vspace*{\baselineskip}
	\includegraphics[width=\textwidth]{copepod_consumption_rate}
	
	\vskip0pt plus 1filll
	
	\tiny\textcopyright\,Levinton, Oxford University Press.
\end{frame}

\begin{frame}[t]{The \highlight{``classic model''} explains GPP at high latitudes.}

	\vspace*{-\baselineskip}
	\begin{center}
		\includegraphics[height=0.82\textheight]{classic_model_gpp}
	\end{center}
	\vskip0pt plus 1filll
	
\tiny\textcopyright\,Levinton, Oxford Univ. Press.
\end{frame}

\begin{frame}[t]{The classic model does not work at tropical latitudes.}

	\vspace*{-0.5\baselineskip}
	\begin{center}
		\includegraphics[width=\textwidth]{global_primary_productivity}
	\end{center}
	\vskip0pt plus 1filll
	
\tiny\textcopyright\,Levinton, Oxford Univ. Press.
\end{frame}

{\usebackgroundtemplate{\includegraphics[width=\paperwidth]{nanoplankton_fossils}}
\begin{frame}[b]
\tiny\hfill\textcolor{white}{Hannes Grobe, Wikimedia Commons}\qquad
\end{frame}
}

\begin{frame}[t]{Bacteria and viruses create a \highlight{microbial loop.}}

	\vspace*{-0.5\baselineskip}
	\begin{center}
		\includegraphics[height=0.82\textheight]{microbial_loop}
	\end{center}
	\vskip0pt plus 1filll
	
\tiny\textcopyright\,Levinton, Oxford Univ. Press.
\end{frame}

\begin{frame}[t]{Polar food webs are relatively simple.}

	\vspace*{-1\baselineskip}
	\begin{center}
		\includegraphics[height=0.82\textheight]{food_web_polar}
	\end{center}
	\vskip0pt plus 1filll
	
\tiny\textcopyright\,Benjamin-Cummings.
\end{frame}

\begin{frame}[t]{Cool temperate food webs are more complex.}

	\vspace*{-1\baselineskip}
	\begin{center}
		\includegraphics[height=0.82\textheight]{food_web_temperate}
	\end{center}
	\vskip0pt plus 1filll
	
\tiny\textcopyright\,Benjamin-Cummings.
\end{frame}

\begin{frame}[t]{Tropical food webs are the most complex.}

	\vspace*{-1\baselineskip}
	\begin{center}
		\includegraphics[height=0.82\textheight]{food_web_tropical}
	\end{center}
	\vskip0pt plus 1filll
	
\tiny\textcopyright\,Benjamin-Cummings.
\end{frame}


\end{document}
