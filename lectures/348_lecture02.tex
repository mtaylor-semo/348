%!TEX TS-program = lualatex
%!TEX encoding = UTF-8 Unicode

\documentclass[t]{beamer}

%%%% HANDOUTS For online Uncomment the following four lines for handout
%\documentclass[t,handout]{beamer}  %Use this for handouts.
%\usepackage{handoutWithNotes}
%\pgfpagesuselayout{3 on 1 with notes}[letterpaper,border shrink=5mm]


%%% Including only some slides for students.
%%% Uncomment the following line. For the slides,
%%% use the labels shown below the command.
%\includeonlylecture{student}

%% For students, use \lecture{student}{student}
%% For mine, use \lecture{instructor}{instructor}


%\usepackage{pgf,pgfpages}
%\pgfpagesuselayout{4 on 1}[letterpaper,border shrink=5mm]

% FONTS
\usepackage{fontspec}
\def\mainfont{Linux Biolinum O}
\setmainfont[Ligatures={Common,TeX}, Contextuals={NoAlternate}, BoldFont={* Bold}, ItalicFont={* Italic}, Numbers={OldStyle, Proportional}]{\mainfont}
\setsansfont[Scale=MatchLowercase, Numbers=OldStyle]{Linux Biolinum O} 
\usepackage{microtype}
%
\usepackage{graphicx}
	\graphicspath{%
	{/Users/goby/Pictures/teach/348/lectures/}%
	{/Users/goby/Pictures/teach/common/}%}%
	{img/}} % set of paths to search for images

\usepackage{amsmath,amssymb}

%\usepackage{units}

%\usepackage{booktabs}
\usepackage{multicol}
%	\setlength{\columnsep=1em}

%\usepackage{textcomp}
%\usepackage{setspace}
\usepackage{tikz}
	\tikzstyle{every picture}+=[remember picture,overlay]

\mode<presentation>
{
  \usetheme{Lecture}
  \setbeamercovered{invisible}
%  \setbeamertemplate{items}[square]
}

%\usepackage{calc}
\usepackage{hyperref}

\newcommand\HiddenWord[1]{%
	\alt<handout>{\rule{\widthof{#1}}{\fboxrule}}{#1}%
}


\begin{document}

\lecture{student}{student}

{
\usebackgroundtemplate{\includegraphics[width=\paperwidth]{how_many_oceans}}
\begin{frame}[b]{How many oceans are on Earth?}
	\hfill \tiny Wikimedia, public domain
\end{frame}
}
%
\lecture{instructor}{instructor}

{\usebackgroundtemplate{\includegraphics[width=\paperwidth]{how_many_oceans}}
\begin{frame}[t]{How many oceans are on Earth?}
	\begin{tikzpicture}
		\node [black, right] at (0.5,-3) {Pacific};
		\node [black, right] at (10.5,-3) {Pacific};
		\node [black, right] at (3.5,-2) {Atlantic};
		\node [black, right] at (7.75,-4) {Indian};
		\node [black, right] at (2.25,-0.1) {Arctic};
		\node [black, right] at (8.5,-0.1) {Arctic};
	\end{tikzpicture}

	\vfilll

	\hfill \tiny Wikimedia, public domain

\end{frame}
}
%
\lecture{student}{student}

{\usebackgroundtemplate{\includegraphics[width=\paperwidth]{southern_ocean}}
\begin{frame}[t]{\null}
	\begin{tikzpicture}
		\node [white, right] at (6,1.1) {Africa};
		\node [white, left] at (2,-3) {S. America};
		\node [white, right] at (10.1,-4) {Australia};
		\node [white, right] at (8,-6.5) {New Zealand};
		\node [black, right] at (5.4,-2.25) {Antarctica};
		\node [white, right] at (5,-0.9) {Southern Ocean};
		\node [white, right] at (5,-5.5) {Southern Ocean};
	\end{tikzpicture}

	\vfilll

	\tiny\textcolor{white}{Google Earth, \copyright\,2015 Google Inc.}

\end{frame}
}
%
{\usebackgroundtemplate{\includegraphics[width=\paperwidth]{deepest_points}}
\begin{frame}[b]{\textcolor{white}{What are the deepest points in the oceans?}}

\pause
	\begin{tikzpicture}[form/.style={circle, inner sep=0pt, minimum size=3mm},
		info/.style={white, align=right}]

		\node[form] (marianas) at (4.7,4.4) [fill=white] {};
		\node[info] [right, align=left] at (marianas.east) {Marianas Trench\\ 10,911m};

\pause
		\node[form] (puerto) at (10,4.6) [fill=white] {};
		\node[info] [right, align=left] at (puerto.east) {Puerto Rico\\ Tr. 8648 m};

\pause
		\node[form] (sunda) at (3.3,3.6) [fill=white] {};
		\node[info] [left] at (sunda.west) {Sunda Trench\\ 7725m};

		\node[form] (diamantina) at (3.3,2.5) [fill=white] {};
		\node[info] [left, align=left] at (diamantina.south west) {Diamantina\\ Deep 8047 m};

\pause
		\node[form] (sandwich) at (11.5,1.3) [fill=white] {};
		\node[info] [left] at (sandwich.west) {Sandwich Trench\\ 7235 m};

\pause
		\node[info] [left] at (12,6) {Litke Deep\\ 5449 m};
		
		\draw [->, white, ultra thick] (11.5,6.5) -- (12.4,7.5);


	\end{tikzpicture}


	\hfill \tiny \textcolor{white}{Seafloor Topography V4.0, \copyright\,W.H.F. Smith \& D.T. Sandwell, 1996.}

\end{frame}
}
%
{\usebackgroundtemplate{\includegraphics[width=\paperwidth]{ocean_tectonics}}
\begin{frame}[b]{Ocean basins are formed by plate tectonics.}
	\hfill \tiny \copyright\,McGraw-Hill
\end{frame}
}
%
\begin{frame}[t]{Earth has \highlight{continental} and \highlight{oceanic} crusts.}

\begin{multicols}{2}

	\includegraphics[width=0.5\textwidth]{two_crust_types}

\columnbreak

	\hangpara Which type is thicker?
	
	\hangpara Which type is older?
	
	\hangpara Which type is denser?
\end{multicols}

	\vfilll

	\hfill \tiny \copyright\,McGraw-Hill

\end{frame}
%
\lecture{instructor}{instructor}

\begin{frame}[t]{Earth has \highlight{continental} and \highlight{oceanic} crusts.}

\begin{multicols}{2}

	\includegraphics[width=0.5\textwidth]{two_crust_types}

\columnbreak

	\hangpara \highlight{Continental} crust

	\hangpara\quad up to 50 km thick\\
			\quad 3.8 billion years old\\
			\quad less dense.
	
	\hangpara \highlight{Oceanic} Crust
	
	\hangpara\quad up to 5 km thick\\
			\quad 200 million years old\\
			\quad more dense.
	
\end{multicols}

	\vfilll

	\hfill \tiny \copyright\,McGraw-Hill

\end{frame}
%
\lecture{student}{student}

{\usebackgroundtemplate{\includegraphics[width=\paperwidth]{ocean_ridges_trenches}}
\begin{frame}[b]{Plate movement creates ridges and trenches.}
	\hfill \tiny \copyright\,Marie Tharp and Bruce Heezen, 1977.
\end{frame}
}
%
\begin{frame}[t]{\highlight{Ridge push:} The weight of the ridge pushes apart the plates.}

\begin{multicols}{2}

	\includegraphics[width=0.5\textwidth]{ridge_push}

\columnbreak

	\hangpara Magma rises, cools, and solidifies.

	\hangpara Plates move 2–18 cm per year.
		
\end{multicols}

	\vfilll

	\hfill \tiny \copyright\,McGraw-Hill

\end{frame}
%
\begin{frame}[t]{\highlight{Slab pull:} oceanic crust sinks at \highlight{subduction zones}, pulling apart the plates.}

	\vspace*{-0.5\baselineskip}

	\begin{multicols}{2}

		\includegraphics[height=0.72\textheight]{slab_pull}

	\columnbreak

		\vspace*{4\baselineskip}
	
		\hangpara Subduction zones create trenches.
	\end{multicols}

	\vfilll

	\hfill \tiny \copyright\,McGraw-Hill

\end{frame}
%
{\usebackgroundtemplate{\includegraphics[width=\paperwidth]{seismic_events}}
\begin{frame}[b]{Seismic events occur at ridges and trenches.}

	\hfill \tiny \textsc{nasa} (\url{http://denali.gsfc.nasa.gov/dtam/seismic/})

\end{frame}
}
%
{\usebackgroundtemplate{\includegraphics[width=\paperwidth]{changing_basins}}
\begin{frame}[b]{Tectonics has rearranged the ocean basins over time.}

	\hfill	\tiny \copyright\,Oxford University Press

\end{frame}
}
%
{\usebackgroundtemplate{\includegraphics[width=\paperwidth]{tethys_sea}}
\begin{frame}[b]{The \highlight{Tethys Sea} was formed about 300 MYA.}

	\hfill \tiny Adrignola, Wikimedia \ccbysa{3}

\end{frame}
}
%
{\usebackgroundtemplate{\includegraphics[width=\paperwidth]{tethys_seaway}}
\begin{frame}[b]{The \highlight{Tethys Seaway} allowed dispersal of tropical organisms among oceans.}

	\hangpara Oceans were connected until 10–15 \textsc{mya} when Africa hit Eurasia, followed by Isthmus of Panama ca. 3 \textsc{mya}.

	\tiny \hfill Modified from \url{scotese.com})

\end{frame}
}
%
{\usebackgroundtemplate{\includegraphics[width=\paperwidth]{oceans_current_configuration}}
\begin{frame}[b]{Tropical organisms in different oceans are isolated by land and temperature.}
\tiny \copyright\,McGraw-Hill)
\end{frame}
}
%
{\usebackgroundtemplate{\includegraphics[width=\paperwidth]{seafloor_sections}}
\begin{frame}[b]{The \highlight{continental margin} consists of the continental shelf, slope and rise.}

	\hfill \tiny \copyright\,McGraw-Hill)

\end{frame}
}
%
{\usebackgroundtemplate{\includegraphics[width=\paperwidth]{seafloor_sections}}
\begin{frame}[b]{The abyssal plain is between 3,000–6,000 m deep.}

	\hfill \tiny \copyright\,McGraw-Hill)

\end{frame}
}
%
{\usebackgroundtemplate{\includegraphics[width=\paperwidth]{global_continental_shelf}}
\begin{frame}[b]{\textcolor{white}{The continental shelf represents about 7–8\% of the ocean surface area.}}

	\hfill \tiny \textcolor{white}{Seafloor Topography \textsc{v}4.0, \copyright\,W.H.F. Smith \& D.T. Sandwell, 1996.}

\end{frame}
}
%
\lecture{instructor}{instructor}
{\usebackgroundtemplate{\includegraphics[width=\paperwidth]{los_angeles_seafloor}}
\begin{frame}[b]{}
	\tiny	\textcolor{white}{\textsc{usgs} Pacific Sea-Floor Mapping Project.}
\end{frame}
}
%
\lecture{instructor}{instructor}

{\usebackgroundtemplate{\includegraphics[width=\paperwidth]{puerto_rico_trench}}
\begin{frame}[b]{}

	\hfill \tiny \textcolor{white}{\textsc{usgs} Project \textsc{probe} Leg II.}

\end{frame}
}
%
\lecture{student}{student}

{\usebackgroundtemplate{\includegraphics[width=\paperwidth]{continental_shelf_na}}
\begin{frame}[b]{}

	\hfill \tiny \textcolor{white}{Google Earth, \copyright\,2015 Google Inc.}

\end{frame}
}
%
{\usebackgroundtemplate{\includegraphics[width=\paperwidth]{continental_shelf_sa}}
\begin{frame}[t]{}

	\vspace*{4\baselineskip}

	\hangpara\hspace*{17em}\parbox[t]{1.5in}{\raggedright\textcolor{white}{Why is the shelf broad on the east coast and narrow on the west coast?}}

	\vspace*{1\baselineskip}

	\hangpara\hspace*{17em}\parbox[t]{1.5in}{\raggedright\textcolor{white}{Will this difference affect marine life? If so, how? If not, why not?}}

	\vfilll

	\tiny \textcolor{white}{Google Earth, \copyright\,2015 Google Inc.}
\end{frame}
}
%
\end{document}
