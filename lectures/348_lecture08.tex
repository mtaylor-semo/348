%!TEX TS-program = lualatex
%!TEX encoding = UTF-8 Unicode

\documentclass[t]{beamer}

%%%% HANDOUTS For online Uncomment the following four lines for handout
%\documentclass[t,handout]{beamer}  %Use this for handouts.
%\includeonlylecture{student}
%\usepackage{handoutWithNotes}
%\pgfpagesuselayout{3 on 1 with notes}[letterpaper,border shrink=5mm]
%	\setbeamercolor{background canvas}{bg=black!5}


%%% Including only some slides for students.
%%% Uncomment the following line. For the slides,
%%% use the labels shown below the command.


%% For students, use \lecture{student}{student}
%% For mine, use \lecture{instructor}{instructor}


%\usepackage{pgf,pgfpages}
%\pgfpagesuselayout{4 on 1}[letterpaper,border shrink=5mm]

% FONTS
\usepackage{fontspec}
\def\mainfont{Linux Biolinum O}
\setmainfont[Ligatures={Common,TeX}, Contextuals={NoAlternate}, BoldFont={* Bold}, ItalicFont={* Italic}, Numbers={Proportional}]{\mainfont}
\setsansfont[Scale=MatchLowercase]{Linux Biolinum O} 
\usepackage{microtype}

\usepackage{graphicx}
	\graphicspath{%
	{/Users/mtaylor/Pictures/teach/348/lectures/}%
	{/Users/mtaylor/Pictures/teach/common/}%}%
	{img/}} % set of paths to search for images

\usepackage{amsmath,amssymb}

%\usepackage{units}

\usepackage{booktabs}
\usepackage{multicol}
%	\setlength{\columnsep=1em}

%\usepackage{tikz}
%	\tikzstyle{every picture}+=[remember picture,overlay]

%\usepackage{chemfig}
%\usepackage[version=4]{mhchem}

%\usepackage{textcomp}
%\usepackage{setspace}
\mode<presentation>
{
  \usetheme{Lecture}
  \setbeamercovered{invisible}
  \setbeamertemplate{items}[square]
}

%\usepackage{animate}

\usepackage{calc} % Needed for \widthof
%\usepackage{hyperref}

\newcommand\HiddenWord[1]{%
	\alt<handout>{\rule{\widthof{#1}}{\fboxrule}}{#1}%
}



\begin{document}

\lecture{student}{student}

{\usebackgroundtemplate{\includegraphics[width=\paperwidth]{biomass_density_depth}}
\begin{frame}[t]{How can the deep sea benthos be so diverse?}

%\vspace*{10\baselineskip}

%\hangpara Yet, the deep sea benthos is very diverse. Why?
\end{frame}}

{
\usebackgroundtemplate{\includegraphics[width=\paperwidth]{deep_sea_floor}}
\begin{frame}[t]{\textcolor{white}{Sanders (1968) proposed the} \textcolor{orange4}{stability-time hypothesis.}} 

	\hangpara \textcolor{white}{What is the importance of stability and time?}

	\hangpara \textcolor{white}{What were the arguments against this hypothesis?}

\tinyfill\textcolor{white}{\href{https://www.aoml.noaa.gov/deep-sea-is-slowing-warming/}{AOML, NOAA, Public Domain.}}

\end{frame}
}
%

%\begin{frame}[t]{\highlight{Stability-time} hypothesis. {\large (Sanders 1968)}}
%
%	\hangpara What is the importance of stability?
%
%	\hangpara What were the arguments against this hypothesis?
%
%\end{frame}

{
\usebackgroundtemplate{\includegraphics[width=\paperwidth]{sea_pigs}}
\begin{frame}[t]{\textcolor{white}{Dayton \& Hessler (1972) proposed the} \textcolor{orange4}{cropper/disturbance hypothesis.}}

\vspace{-1.05\baselineskip}



\textcolor{white}{\rule{650px}{0.4pt}}

\vfilll

\tiny \textcolor{white}{Sea pigs, \textcopyright\,Monterrey Bay Aquariun and Research Institute}
\end{frame}
}
%
%\begin{frame}[t]{\highlight{Cropper/disturbance} hypothesis. {\large (Dayton \& Hessler 1972)}}
%
%	\hangpara What was the reasoning for this hypothesis?
%
%	\hangpara What were the arguments against this hypothesis?
%
%	\hangpara What are r- and K- life history strategies?\\ Why should predation lead to an r-strategy?\\ Why would most deep-sea organisms have a K-strategy?
%
%\end{frame}

{
\usebackgroundtemplate{\includegraphics[width=\paperwidth]{deep_sea_floor1}}
\begin{frame}[t]{\textcolor{white}{Abele \& Walters proposed the }\textcolor{orange4}{species-area hypothesis.} }

	\hangpara \textcolor{white}{What was the reasoning for this hypothesis?}

	\hangpara \textcolor{white}{What was the major argument against this hypothesis?}
	
	\tinyfill Amon~et~al.~2016. Sci.\ Rep.\ 6:30492. \ccbysa{2.0}

\end{frame}
}

%\begin{frame}[t]{\highlight{Species-area} hypothesis. {\large (Abele \& Walters 1979)}}
%
%	\hangpara What was the reasoning for this hypothesis?
%
%	\hangpara What was the major argument against this hypothesis?
%
%\end{frame}

{
\usebackgroundtemplate{\includegraphics[width=\paperwidth]{flounder_swimming}}
\begin{frame}[t]{\textcolor{white}{Grassle \& Sanders (1973) proposed the} \textcolor{orange4}{{patch dynamics hypothesis.}}}

\vspace{-1\baselineskip}

\textcolor{gray}{\rule{\textwidth}{0.4pt}}

	%\hangpara What do you think is meant by a patchy habitat?

	%\hangpara What do you think is meant by dynamics?
	
	%\hangpara How could this explain diversity in the deep sea?

\tinyfill \href{https://www.visitpensacolabeach.com/ecotrail/fish/bottom-fish/}{\textcopyright\,VisitPensacolaBeach.com}
\end{frame}
}

%{
%\usebackgroundtemplate{\includegraphics[width=\paperwidth]{flounder_swimming}}
%\begin{frame}[t]{\highlight{Patch dynamics} hypothesis. {\large (Grassle \& Sanders 1973)}}
%
%	\hangpara What do you think is meant by a patchy habitat?
%
%	\hangpara What do you think is meant by dynamics?
%	
%	\hangpara How could this explain diversity in the deep sea?
%
%\end{frame}
%}

\begin{frame}[t]{\highlight{Spatial heterogeneity} means the habitat is not uniform.}

	{\centering
	\includegraphics[width=0.75\textwidth]{spatial_heterogeneity_abyssal_plain}\par}

	\vspace*{-\baselineskip}\hangpara North Atlantic abyssal plain (4800 m depth). Translucent sea cucumbers are feeding on green detritus from surface.

\vskip0pt plus 1filll

\hfill\tiny\url{http://www.oceanlab.abdn.ac.uk/esonet/porcupine.php}
\end{frame}


{\usebackgroundtemplate{\includegraphics[width=\paperwidth]{deep_sea_benthic_diversity}}
\begin{frame}[b]

\hfill\tiny After Rex 1981, Annu. Rev. Ecol. Syst. 122: 331.
\end{frame}}



%\begin{frame}[t]{Develop a hypothesis to explain diversity at intermediate depths.}
%
%	{\centering
%	\includegraphics[width=0.90\textwidth]{deep_sea_benthic_intermediate_diversity}\par}
%
%	\vspace*{-\baselineskip}
%	\hangpara Use a combination of previous hypotheses and your own ideas.
%	
%	\hangpara Consider biomass, age, stability, competition, predation, sediment size, disturbance and patch dynamics, spatial heterogeneity, etc.
%
%\vskip0pt plus 1filll
%
%\hfill\tiny After Rex 1981, Annu. Rev. Ecol. Syst. 122: 331.
%\end{frame}

\begin{frame}[t]{Pull out a sheet of paper. In groups\dots}

	\vspace*{-\baselineskip}

	\begin{multicols}{2}

	\includegraphics[width=0.49\textwidth]{deep_sea_benthic_intermediate_diversity2}

	\columnbreak
	
	\hangpara Choose three factors from previous hypotheses or your own ideas.
	
	\hangpara Consider biomass, age, stability, competition, predation, sediment size, disturbance, patch dynamics, spatial heterogeneity, etc.

	\hangpara Develop a unique hypothesis by explaining how each factor contributes to benthic diversity. 
	
	\hangpara Explain how each factor contributes to diversity. 

	\end{multicols}

\end{frame}


\begin{frame}[t]{What is the scientific consensus?}

\hangpara \rule{2.5in}{0.4pt}

\vspace*{2\baselineskip}

\hangpara \rule{2.5in}{0.4pt}

\vspace*{2\baselineskip}

\hangpara \rule{2.5in}{0.4pt}

\end{frame}




\lecture{instructor}{instructor}

\begin{frame}[t]{What is the scientific consensus?}

\hangpara\highlight{Spatial heterogeneity}\pause

\vspace*{2\baselineskip}

\hangpara\highlight{Patch dynamics}\pause

\vspace*{2\baselineskip}

\hangpara\highlight{Sediment size}

\end{frame}



\lecture{student}{student}
\begin{frame}[t]{Many mesopelagic organisms make a \highlight{diel vertical migration.}}

	{\centering
	\includegraphics[width=\textwidth]{diel_vertical_migration}\par}

\end{frame}


\begin{frame}[t]{Can nutrients that reach the deep sea be returned to the surface?}
\end{frame}


{\usebackgroundtemplate{\includegraphics[width=\paperwidth]{upwelling_global}}
\begin{frame}[b]{\highlight{Upwelling} returns nutrients to surface.}

\hfill\tiny\textcopyright\,McGraw-Hill
\end{frame}}


\begin{frame}[t]{Upwelling is driven by winds and Ekman transport.}

{\centering
\includegraphics[height=0.80\textheight]{upwelling_equatorial}\par}

\vskip0pt plus 1filll

\hfill\tiny\textcopyright\,McGraw-Hill
\end{frame}



\end{document}
