%!TEX TS-program = lualatex
%!TEX encoding = UTF-8 Unicode

\documentclass[t]{beamer}

%%%% HANDOUTS For online Uncomment the following four lines for handout
%\documentclass[t,handout]{beamer}  %Use this for handouts.
%\includeonlylecture{student}
%\usepackage{handoutWithNotes}
%\pgfpagesuselayout{3 on 1 with notes}[letterpaper,border shrink=5mm]


%% For students, use \lecture{student}{student}
%% For mine, use \lecture{instructor}{instructor}

% FONTS
\usepackage{fontspec}
\def\mainfont{Linux Biolinum O}
\setmainfont[Ligatures={Common,TeX}, Contextuals={NoAlternate}, BoldFont={* Bold}, ItalicFont={* Italic}, Numbers={Proportional}]{\mainfont}
\setsansfont[Scale=MatchLowercase]{Linux Biolinum O} 
\usepackage{microtype}

\usepackage{graphicx}
	\graphicspath{%
	{/Users/goby/Pictures/teach/348/lectures/}%
	{/Users/goby/Pictures/teach/common/}%}%
	{img/}} % set of paths to search for images

\usepackage{amsmath,amssymb}

%\usepackage{units}

\usepackage{booktabs}
\usepackage{multicol}
%	\setlength{\columnsep=1em}

%\usepackage{tikz}
%	\tikzstyle{every picture}+=[remember picture,overlay]

%\usepackage{chemfig}
%\usepackage[version=4]{mhchem}

%\usepackage{textcomp}
%\usepackage{setspace}
\mode<presentation>
{
  \usetheme{Lecture}
  \setbeamercovered{invisible}
  \setbeamertemplate{items}[square]
}

%\usepackage{animate}

\usepackage{calc} % Needed for \widthof
%\usepackage{hyperref}

\newcommand\HiddenWord[1]{%
	\alt<handout>{\rule{\widthof{#1}}{\fboxrule}}{#1}%
}



\begin{document}

\lecture{student}{student}

\begin{frame}[t]{Ridde me this, Nemo.}

	\hangpara Read pages 414--417 and handout.

	\hangpara Discuss the relative merits of the three hypotheses that attempt to explain the relatively high species diversity of the deep-sea benthos.

	\hangpara Typed for next lecture (no late acceptances).
	
\end{frame}

{\usebackgroundtemplate{\includegraphics[width=\paperwidth]{cave_vs_bathypelagic_fishes}}
\begin{frame}[b]{Why are cave fishes white but bathypelagic fishes black?}

\tiny Hoosier cave fish, Prosanta Chakrabarty, Wikimedia Commons.\hfill Gulper eel, Alexei Orlvo, Wikimedia Commons.
\end{frame}}

{\usebackgroundtemplate{\includegraphics[width=\paperwidth]{epipelagic_nutrient_sources}}
\begin{frame}[t]{How much NPP reaches deep sea from epipelagic?}
\end{frame}}

\begin{frame}[t]{Whale fall is a staged source of nutrients.}

\vspace*{-\baselineskip}

	\begin{multicols}{2}
		{\centering
		\includegraphics[height=0.82\textheight]{whale_fall}\par}

	\columnbreak

		\hangpara Mobile-scavenger stage (months to years)

		\hangpara Enrichment opportunist stage (months to years)

		\hangpara Sulphophilic stage (decades)

		\hangpara Reef stage (overlaps other stages)

	\end{multicols}

%\vskip0pt plus 1filll

\end{frame}

{\usebackgroundtemplate{\includegraphics[width=\paperwidth]{whale_fall_enriched_reef}}
\begin{frame}[b]

\tiny\hfill\textcolor{white}{Craig Smith NOAA, Wikimedia Commons.}
\end{frame}}

{\usebackgroundtemplate{\includegraphics[width=\paperwidth]{boneworm}}
\begin{frame}[b]

\tiny\textcolor{white}{Robert Vrijenhoek, Creative Commons, \url{http://chess.myspecies.info/}.\hfill\href{https://www.youtube.com/watch?v=rdI3eFrTGs8}{Link to Video}}

\end{frame}}

{\usebackgroundtemplate{\includegraphics[width=\paperwidth]{boneworm_roots}}
\begin{frame}[b]{Boneworms have heterotrophic symbiotic bacteria that decompose bone.}


\tiny\textcopyright\,Levinton, Oxford Press.
\end{frame}}


{\usebackgroundtemplate{\includegraphics[width=\paperwidth]{hydrothermal_vent_black_smoker}}
\begin{frame}[t]

\vspace*{3\baselineskip}

\hspace*{65mm}\hangpara\parbox{49mm}{\raggedright \large\textcolor{white}{Seafloor spreading creates hydrothermal vents.\\[\baselineskip]
Super-heated water has reduced compounds.}}

\vskip0pt plus 1filll

\tiny\textcolor{white}{Ocean Networks Canada, Flickr Creative Commons.}
\end{frame}}

{\usebackgroundtemplate{\includegraphics[width=\paperwidth]{hydrothermal_vents_global_distribution}}
\begin{frame}[b]{Hydrothermal vents and cold seeps are sources of \textsc{npp}.}

\tiny\hfill German et al. 2011. PLoS ONE 6(8): e23259.
\end{frame}}



{\usebackgroundtemplate{\includegraphics[width=\paperwidth]{chemosynthesis}}
\begin{frame}[b]{Bacterial chemosynthesis oxidizes reduced compounds to release energy.}

\tiny\textcopyright\,Pearson Education.
\end{frame}}

\begin{frame}[t]{Bacteria are primary producers for vent and cold seep communities.}

\vspace*{-\baselineskip}

	\begin{multicols}{2}
		{\centering
		\includegraphics[height=0.82\textheight]{vent_community_mosaic}\par}

	\columnbreak

		\hangpara Free-living bacteria

		\hangpara Symbiotic bacteria

	\end{multicols}

\end{frame}


{\usebackgroundtemplate{\includegraphics[width=\paperwidth]{vent_community}}
\begin{frame}[b]{\textcolor{white}{NPP supports rich vent community.}}

\tiny\hfill\textcolor{white}{Ocean Networks Canada, Flickr Creative Commons.}
\end{frame}}

{\usebackgroundtemplate{\includegraphics[width=\paperwidth]{cold_seep_community}}
\begin{frame}[b]{\textcolor{white}{NPP supports rich cold seep community.}}

\tiny\textcolor{white}{NOAA Photo Library, Flickr Creative Commons.}
\end{frame}}

{\usebackgroundtemplate{\includegraphics[width=\paperwidth]{vent_community_dead}}
\begin{frame}[b]{\textcolor{white}{Vent communities last only a few hundred years.}}

\tiny\textcolor{white}{Ocean Networks Canada, Flickr Creative Commons.}
\end{frame}}

{\usebackgroundtemplate{\includegraphics[width=\paperwidth]{sediment_size}}
\begin{frame}[b]{Sediment size affects the benthic community.}

\tiny\textcopyright\,McGraw-Hill
\end{frame}}


\begin{frame}[t]{What about other deep sea benthic communities?}

\vspace*{-\baselineskip}

	\begin{multicols}{2}
		{\centering
		\includegraphics[width=0.49\textwidth]{tripod_fish}
		
		\vspace{0.5\baselineskip}
	
		\includegraphics[width=0.49\textwidth]{deep_sea_cucumbers}
		\par}

	\columnbreak

		\hangpara What type of sediment is present?

		\hangpara Is the sediment nutrient rich or poor?
		
		\hangpara What type of infauna is present?
		
		\hangpara What is their feeding mechanism?

	\end{multicols}

\end{frame}

\begin{frame}[t]{Consider what you know about marine snow.}

\hangpara How do you think the vertical distribution of nutrients from the surface downward affects the vertical distribution of biomass and density of individuals?

\end{frame}

{\usebackgroundtemplate{\includegraphics[width=\paperwidth]{biomass_density_depth_student}}
\begin{frame}[t]{Plot biomass and density as a function of depth.}

\end{frame}}

\lecture{instructor}{instructor}

{\usebackgroundtemplate{\includegraphics[width=\paperwidth]{biomass_density_depth}}
\begin{frame}[t]{Both biomass and density decrease with depth.}

\vspace*{10\baselineskip}

\hangpara Yet, the deep sea benthos is very diverse. Why?
\end{frame}}



\end{document}
