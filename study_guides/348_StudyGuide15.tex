%!TEX TS-program = lualatex
%!TEX encoding = UTF-8 Unicode

\documentclass[nofonts, letterpaper]{tufte-handout}

%\geometry{showframe} % display margins for debugging page layout

\usepackage{graphicx} % allow embedded images
  \setkeys{Gin}{width=\linewidth,totalheight=\textheight,keepaspectratio}
  \graphicspath{{img/}} % set of paths to search for images
  
\usepackage{fontspec}
  \setmainfont[Ligatures={Common,TeX},Numbers={Proportional}]{Linux Libertine O}
  \setsansfont{Linux Biolinum O}
\usepackage{microtype}
\usepackage{enumitem}
\usepackage{multicol} % multiple column layout facilities
\usepackage[version=4]{mhchem}
%\usepackage{hyperref}
%\usepackage{fancyvrb} % extended verbatim environments
%  \fvset{fontsize=\normalsize}% default font size for fancy-verbatim environments

% Change the header to shift the title to the left side of the page. 
% Replaced \quad with \hfill.  See \plaintitle in tufte-common.def
{\fancyhead[RE,RO]{\scshape{\newlinetospace{\plaintitle}}\hfill\thepage}}

\makeatletter
% Paragraph indentation and separation for normal text
\renewcommand{\@tufte@reset@par}{%
  \setlength{\RaggedRightParindent}{1.0pc}%
  \setlength{\JustifyingParindent}{1.0pc}%
  \setlength{\parindent}{1pc}%
  \setlength{\parskip}{0pt}%
}
\@tufte@reset@par

% Paragraph indentation and separation for marginal text
\renewcommand{\@tufte@margin@par}{%
  \setlength{\RaggedRightParindent}{0pt}%
  \setlength{\JustifyingParindent}{0.5pc}%
  \setlength{\parindent}{0.5pc}%
  \setlength{\parskip}{0pt}%
}

\makeatother

% Set up the spacing using fontspec features
\renewcommand\allcapsspacing[1]{{\addfontfeatures{LetterSpace=15}#1}}
\renewcommand\smallcapsspacing[1]{{\addfontfeatures{LetterSpace=10}#1}}

\title{Study Guide 15}%\hfill}
\author{Ocean Acidification: Effects}

\date{} % without \date command, current date is supplied

\begin{document}

\maketitle	% this prints the handout title, author, and date

%\printclassoptions

%\section{Vocabulary}
%\vspace{-1\baselineskip}
%\begin{multicols}{2}
%aragonite \\
%fecundity 
%\end{multicols}

\section{Concepts}
\marginnote{\textbf{Study:} pgs. 32--27, 271--272, 448--450, 582--584, plus Hoegh-Guldberg et al. 2007, available on the course website.}

\begin{enumerate}

\item How have aragonite levels changed at depth in the ocean since pre-industrial times. Explain how this change relates to pH change in the oceans.

\item How does increased atmospheric \ce{CO2} affect the ability of corals to produce skeletons? How does this affect the fitness of these corals?

\item Describe how \ce{CO2} increases affects some of the major groups of organisms (e.g., coccolithphores, molluscs) in their ability for calcification, photosynthesis, and reproduction. Can you explain why?

\item Explain how acidification affects different stages in the life cycle of benthic calcifers like sea urchins.

\item Describe how increased atmospheric \ce{CO2} affects the ability of some fishes to identify suitable habitat and potential predators.

\end{enumerate}


\end{document}