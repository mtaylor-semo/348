%!TEX TS-program = lualatex
%!TEX encoding = UTF-8 Unicode

\documentclass[letterpaper]{tufte-handout}

%\geometry{showframe} % display margins for debugging page layout

\usepackage{graphicx} % allow embedded images
  \setkeys{Gin}{width=\linewidth,totalheight=\textheight,keepaspectratio}
  \graphicspath{{img/}} % set of paths to search for images
  
\usepackage{fontspec}
  \setmainfont[Ligatures={Common,TeX},Numbers={Proportional}]{Linux Libertine O}
  \setsansfont{Linux Biolinum O}
\usepackage{microtype}
\usepackage{enumitem}
\usepackage{multicol} % multiple column layout facilities
%\usepackage{hyperref}
%\usepackage{fancyvrb} % extended verbatim environments
%  \fvset{fontsize=\normalsize}% default font size for fancy-verbatim environments

% Change the header to shift the title to the left side of the page. 
% Replaced \quad with \hfill.  See \plaintitle in tufte-common.def
{\fancyhead[RE,RO]{\scshape{\newlinetospace{\plaintitle}}\hfill\thepage}}

\makeatletter
% Paragraph indentation and separation for normal text
\renewcommand{\@tufte@reset@par}{%
  \setlength{\RaggedRightParindent}{1.0pc}%
  \setlength{\JustifyingParindent}{1.0pc}%
  \setlength{\parindent}{1pc}%
  \setlength{\parskip}{0pt}%
}
\@tufte@reset@par

% Paragraph indentation and separation for marginal text
\renewcommand{\@tufte@margin@par}{%
  \setlength{\RaggedRightParindent}{0pt}%
  \setlength{\JustifyingParindent}{0.5pc}%
  \setlength{\parindent}{0.5pc}%
  \setlength{\parskip}{0pt}%
}
\makeatother

\title{Study Guides 01 and 02\hfill}
\author{Ocean Formation and Plate Tectonics}

\date{} % without \date command, current date is supplied

\begin{document}

\maketitle	% this prints the handout title, author, and date

%\printclassoptions

\section{Using the Study Guides}
The\marginnote{\textbf{Read:} pgs. 3--4, 9--13, 17--22. You should be able to apply the scientific method and hypothesis testing.\\\textbf{Questions:} pg. 16: 2, 4. I will not collect your answers to the questions listed in the study guide but you should answer them. They may be used at the bases for exam or homework questions.} study guides will help you learn the material.  Each study guide contains vocabulary to learn and a series of questions based on the lectures and the assigned reading from the textbook.  The guides may also contain more information to supplement the lecture.  Read the study guides in advance of lecture to get familiar with the day's topic. Bring the study guide to class to see the vocabulary and questions in context of the lecture discussion.  This will help you recall the information during your daily (or every other day) study.

\section{Vocabulary}
\vspace{-1\baselineskip}
\begin{multicols}{2}
oceans\\
ocean basin\\
salinity \\
plate tectonics\\
continental crust\\
oceanic crust\\
mid-ocean ridge\\
ridge push\\
subduction zone\\
slab pull\\
Panthalassa\\
Tethys Sea\\
Tethys Seaway\\
continental margin\\
continental shelf\\
continental slope\\
continental rise\\
shelf break\\
abyssal plain\\
\end{multicols}

The vocabulary lists the terms from each lecture that you should know. You must be able to recognize and apply these terms in a broader context.  I will use the terms in lecture and on exams. If you do not know the terms, you may not fully understand the lecture or be able to answer a question on the exam. I expect you to use the proper vocabulary in your answers to questions on exams and assignments.  Get in the habitat of using terms as you learn the material.  I may not cover all terms in class or do so only in passing.  I expect that you will learn them by reading your textbook and using the glossary in your textbook, to put them into the context of organismal biology.

\section{Concepts}

You should \emph{write} clear and concise answers to each question in the Concepts section.  The questions are not necessarily independent.  Think broadly across lectures to see ``the big picture.''  Study guide questions may be used as a basis for short answer or essay questions on the exam. I may also create exam and assignment questions that do not appear on the study guides. These are guides, not exhaustive test banks.

\begin{enumerate}
	\item Define broadly the marine environment. What is the minimum salinity requirement for a habitat to be considered ``marine.''  What is the average salinity of the open ocean?

\item What are the most important chemical components of seawater? (Those compounds that contribute 1\% or more of total salinity).

\item How did the ocean get ``salty''? Is it getting saltier?  Explain.

\item How many oceans are there?  Name them. What about ocean basins?  If the numbers do not equal, what explains the difference? Be able to identify or mark all of the oceans and corresponding basins on a map.

\item List one or two reasons why the Southern Ocean is recognized as a distinct ocean.

\item Why is the ``plate tectonics'' a more accurate name than ``continental drift'' for the geological theory of moving continents?

\item Briefly describe the difference between oceanic and continental crusts.  Specifically address age, thickness and relative density.

\item Explain slab pull and ridge push. Explain how each contributes to plate tectonics.

\item What is the importance of the mid-ocean ridge system?  What is the importance of the trenches?  That is, how do they contribute to our understanding of plate tectonics and the formations of oceans?

\item Explain how the tectonic plates move, according to the Theory of Plate Tectonics.  Include the role of subduction and mid-ocean ridges.

\item Why are volcanos and earthquakes common occurrences in regions where oceanic trenches are located?  Explain the likely cause of these earthquakes and volcanos.

\item Which is older, continental crust or oceanic crust, and why?

\item What is the importance of the Tethys Seaway?  Is the Tethys Sea and the Tethys Seaway the same thing? Why or why not?

\item What is Panthalassa? Is it the same as the Tethys Sea or Tethys Seaway?  Why or why not?

\item Describe a typical ocean basin.  Include the different parts of the sea floor and how they relate to each other.   

\item Describe a typical ocean basin as a transect across the ocean floor from one continent to another.  For example, what if you traveled across the sea floor from Maine to Spain?  What major features of the sea floor would you be likely to find?  What if you traveled across the sea floor from California to Japan? How would that differ from your trip across the Atlantic sea floor?  Use Google Earth or look at a Tharp map: 

\url{http://www.s-yamaga.jp/nanimono/chikyu/TharpMapLarge.jpg}

\item As a group exercise, we discussed why the eastern continental shelfs of North and South America are broad but the western continental shelfs are narrow.  Why was this?  How could these differences affect marine life? (Hint: Would the type of substrate, such as sand, mud, or rock, differ?  Would the type of substrate affect the organisms living there?)  

\end{enumerate}


\end{document}