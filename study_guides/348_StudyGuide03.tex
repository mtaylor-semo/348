%!TEX TS-program = lualatex
%!TEX encoding = UTF-8 Unicode

\documentclass[nofonts, letterpaper]{tufte-handout}

%\geometry{showframe} % display margins for debugging page layout

\usepackage{graphicx} % allow embedded images
  \setkeys{Gin}{width=\linewidth,totalheight=\textheight,keepaspectratio}
  \graphicspath{{img/}} % set of paths to search for images
  
\usepackage{fontspec}
  \setmainfont[Ligatures={Common,TeX},Numbers={Proportional}]{Linux Libertine O}
  \setsansfont{Linux Biolinum O}
\usepackage{microtype}
\usepackage{enumitem}
\usepackage{multicol} % multiple column layout facilities
%\usepackage{hyperref}
%\usepackage{fancyvrb} % extended verbatim environments
%  \fvset{fontsize=\normalsize}% default font size for fancy-verbatim environments

% Change the header to shift the title to the left side of the page. 
% Replaced \quad with \hfill.  See \plaintitle in tufte-common.def
{\fancyhead[RE,RO]{\scshape{\newlinetospace{\plaintitle}}\hfill\thepage}}

\makeatletter
% Paragraph indentation and separation for normal text
\renewcommand{\@tufte@reset@par}{%
  \setlength{\RaggedRightParindent}{1.0pc}%
  \setlength{\JustifyingParindent}{1.0pc}%
  \setlength{\parindent}{1pc}%
  \setlength{\parskip}{0pt}%
}
\@tufte@reset@par

% Paragraph indentation and separation for marginal text
\renewcommand{\@tufte@margin@par}{%
  \setlength{\RaggedRightParindent}{0pt}%
  \setlength{\JustifyingParindent}{0.5pc}%
  \setlength{\parindent}{0.5pc}%
  \setlength{\parskip}{0pt}%
}

\makeatother

% Set up the spacing using fontspec features
\renewcommand\allcapsspacing[1]{{\addfontfeatures{LetterSpace=15}#1}}
\renewcommand\smallcapsspacing[1]{{\addfontfeatures{LetterSpace=10}#1}}


\title{Study Guide 03}
\author{Ocean Circulation and Larval Ecology}

\date{} % without \date command, current date is supplied

\begin{document}

\maketitle	% this prints the handout title, author, and date

%\printclassoptions

\section{Vocabulary}\marginnote{\textbf{Study:} pgs. 22--23, 91–93, 118–122, 133--138, 141--145. Review basic ecology principles on 40–56.\\\noindent\textbf{Questions:} pg.~101: 2, 4; pg.~139: 9, 10, 12–14.}
\vspace{-1\baselineskip}
\begin{multicols}{2}
neritic \\
oceanic \\
pelagic \\
epipelagic \\
mesopelagic \\
bathypelagic \\
photic zone \\
disphotic zone\\
aphotic zone\\
littoral \\
intertidal \\
sublittoral \\
benthic \\
bathyal \\
abyssal \\
gyre \\
Coriolis Effect\\
Ekman transport\\
Reynolds number\\
larvae (singular: larva) \\
planktotrophic larvae \\
lecithotrophic larvae \\
nonpelagic larvae 
\end{multicols}

\section{Concepts}

\begin{enumerate}

\item
  Know the basic zones of the ocean. They will form the framework of our
  community discussions.

\item
  If you were to look at a band of tropical water (surface temperature
  of 25°C and higher) across an ocean basin, the band would be wider
  along the west side of the basin and narrower along the east side of
  the basin. What explains this phenomenon?

\item
  Are the photic, disphotic, and aphotic zones of the ocean synonyms for
  other names of the oceans zones? For example, is the photic zone
  synonymous with the epipelagic zone? Are some of the zones
  functionally equivalent? Explain.
  
\item Describe how the global wind patterns are formed, based on thermal input from the sun and from the Coriolis effect. You do have to explain what causes the Coriolis effect itself.

\item Describe how the winds and Coriolis effect combine to form the major oceanic current gyres, as well as major surface wind patterns.

\item Why does water move at a right angle (90°) relative to the direction of wind? What is this phenomenon called?

\item List and briefly explain some reasons why ocean circulation (surface currents) matters to the biology of some marine organisms.

\item Explain the relationship between the Reynolds number, organism size, and the ease with which an organism can move through water.

\item
  The density of water creates a new type of community not found in
  terrestrial ecosystems. What is this community? What types of
  adaptations evolved in other communities to take advantage of this new
  community? (Hint: think trophic). Provide several examples of the
  adaptations.

\item
  Compare and contrast the three types of larval development. Consider
  larval dispersal ability, larval mortality, clutch size (number of
  offspring produced), species distribution, and population genetic
  structure of the species. Relate population genetic structure to the
  potential for speciation to occur (this will take some thought).

 \item
  Compare and contrast the three types of larval development. Consider
  larval dispersal ability, larval mortality, clutch size (number of
  offspring produced), species distribution, and population genetic
  structure of the species. Relate population genetic structure to the
  potential for speciation to occur (this will take some thought).

\item At what latitudes (speaking generally) did the lecithotrophic larval strategy become more common for many marine organisms?  Why?

\end{enumerate}


\end{document}