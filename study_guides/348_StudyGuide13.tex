%!TEX TS-program = lualatex
%!TEX encoding = UTF-8 Unicode

\documentclass[nofonts, letterpaper]{tufte-handout}

%\geometry{showframe} % display margins for debugging page layout

\usepackage{graphicx} % allow embedded images
  \setkeys{Gin}{width=\linewidth,totalheight=\textheight,keepaspectratio}
  \graphicspath{{img/}} % set of paths to search for images
  
\usepackage{fontspec}
  \setmainfont[Ligatures={Common,TeX},Numbers={Proportional}]{Linux Libertine O}
  \setsansfont{Linux Biolinum O}
\usepackage{microtype}
\usepackage{enumitem}
\usepackage{multicol} % multiple column layout facilities
%\usepackage{hyperref}
%\usepackage{fancyvrb} % extended verbatim environments
%  \fvset{fontsize=\normalsize}% default font size for fancy-verbatim environments

% Change the header to shift the title to the left side of the page. 
% Replaced \quad with \hfill.  See \plaintitle in tufte-common.def
{\fancyhead[RE,RO]{\scshape{\newlinetospace{\plaintitle}}\hfill\thepage}}

\makeatletter
% Paragraph indentation and separation for normal text
\renewcommand{\@tufte@reset@par}{%
  \setlength{\RaggedRightParindent}{1.0pc}%
  \setlength{\JustifyingParindent}{1.0pc}%
  \setlength{\parindent}{1pc}%
  \setlength{\parskip}{0pt}%
}
\@tufte@reset@par

% Paragraph indentation and separation for marginal text
\renewcommand{\@tufte@margin@par}{%
  \setlength{\RaggedRightParindent}{0pt}%
  \setlength{\JustifyingParindent}{0.5pc}%
  \setlength{\parindent}{0.5pc}%
  \setlength{\parskip}{0pt}%
}

\makeatother

% Set up the spacing using fontspec features
\renewcommand\allcapsspacing[1]{{\addfontfeatures{LetterSpace=15}#1}}
\renewcommand\smallcapsspacing[1]{{\addfontfeatures{LetterSpace=10}#1}}

\title{Study Guide 13}%\hfill}
\author{Salt Marshes}

\date{} % without \date command, current date is supplied

\begin{document}

\maketitle	% this prints the handout title, author, and date

%\printclassoptions

\section{Vocabulary}
\marginnote{\textbf{Study:} pgs. 385--400, plus salt marsh handout given in class and available online.}
\vspace{-1\baselineskip}
\begin{multicols}{2}
low marsh \\
high marsh \\
creek \\
salt pan \\
wrack \\
cordgrass \\
salt meadow hay \\
black rush \\
marsh elder
\end{multicols}

\section{Concepts}

\begin{enumerate}
\item
  You should know the four major marsh plants above as they define the
  zones of a typical New England salt marsh.
\item
  Explain the distinction between the low marsh and the high marsh.
\item
  What are the two primary types of interactions that, taken together,
  explain zonation in a New England salt marsh?
\item
  Explain how ice and wrack and create patches in a a northern marsh.
  Why are they important for patch creation in northern marshes but not
  generally in southern marshes?
\item
  Explain patch dynamics in a salt marsh. Include the role of patch size
  and the physical stress of a patch, and whether the patch forms in the
  low marsh or the high marsh. Compare and contrast patch dynamics in a
  salt marsh with patch dynamics in the deep sea and the rocky
  intertidal. How do they differ in successional patterns? How are they
  similar?
\item
  Explain why salinity has a greater impact in southern marshes compared
  to northern marshes? Also consider large bare patches in a northern
  marsh compared to the surrounding vegetated areas.
\item
  Explain how physical and competitive forces interact to explain
  zonation of the four major marsh plant species in a New England marsh.
\item
  Explain how positive interactions between marsh species also
  contribute to zonation in a New England salt marsh.
\item
  Why is only cordgrass found in the low marsh?
\item
  How do predators influence the distribution of prey species in a
  marsh?
\item
  Explain the physical and biotic factors that influence the gradient
  distribution of oxygen in the sediment of a salt marsh.
\item
  Explain how salt-tolerant species of plants can facilitate the
  recolonization of a patch by the dominant plant species.
\end{enumerate}


\end{document}