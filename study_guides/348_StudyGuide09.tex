%!TEX TS-program = lualatex
%!TEX encoding = UTF-8 Unicode

\documentclass[nofonts, letterpaper]{tufte-handout}

%\geometry{showframe} % display margins for debugging page layout

\usepackage{graphicx} % allow embedded images
  \setkeys{Gin}{width=\linewidth,totalheight=\textheight,keepaspectratio}
  \graphicspath{{img/}} % set of paths to search for images
  
\usepackage{fontspec}
  \setmainfont[Ligatures={Common,TeX},Numbers={Proportional}]{Linux Libertine O}
  \setsansfont{Linux Biolinum O}
\usepackage{microtype}
\usepackage{enumitem}
\usepackage{multicol} % multiple column layout facilities
%\usepackage{hyperref}
%\usepackage{fancyvrb} % extended verbatim environments
%  \fvset{fontsize=\normalsize}% default font size for fancy-verbatim environments

% Change the header to shift the title to the left side of the page. 
% Replaced \quad with \hfill.  See \plaintitle in tufte-common.def
{\fancyhead[RE,RO]{\scshape{\newlinetospace{\plaintitle}}\hfill\thepage}}

\makeatletter
% Paragraph indentation and separation for normal text
\renewcommand{\@tufte@reset@par}{%
  \setlength{\RaggedRightParindent}{1.0pc}%
  \setlength{\JustifyingParindent}{1.0pc}%
  \setlength{\parindent}{1pc}%
  \setlength{\parskip}{0pt}%
}
\@tufte@reset@par

% Paragraph indentation and separation for marginal text
\renewcommand{\@tufte@margin@par}{%
  \setlength{\RaggedRightParindent}{0pt}%
  \setlength{\JustifyingParindent}{0.5pc}%
  \setlength{\parindent}{0.5pc}%
  \setlength{\parskip}{0pt}%
}

\makeatother

\title{Study Guide 09}
\author{Rocky Sublittoral Communities}

\date{} % without \date command, current date is supplied

\begin{document}

\maketitle	% this prints the handout title, author, and date

%\printclassoptions

\section{Vocabulary}
\marginnote{\textbf{Study:} pgs. 38, 418--432.}
\vspace{-1\baselineskip}
\begin{multicols}{2}
wave crest\\
wave trough\\
wave length\\
wave height\\
rogue waves\\
tsunami\\
subtidal \\
sublittoral \\
infralittoral \\
circalittoral \\
keystone species \\
trophic cascade \\
alternate stable states
\end{multicols}

\section{Concepts}

\begin{enumerate}

\item 
  A friend tells you that waves are due to water moving from 
  offshore to onshore.  Explain to your friend what forms ocean 
  waves and what really moves through the water. Describe some 
  observational evidence to back up your explanation.

\item 
  Your friend counters your explanation by saying that water has to 
  be moving forward in wave because otherwise surfers wouldn't 
  be able to surf.  Expand your explanation of waves to explain the 
  forces that propel surfers forward to the shore.\sidenote{Or the rocks if they are not good surfers}

\item
  What is a rogue wave? What hypothesis has been put forward 
  to explain the presence of rogue waves.

\item
  Now your friend tries to tell you that tsunami are caused by freak 
  tides. (``Dude, why else would they be called tidal waves?'')
  \sidenote{You really should pick a better class of friends!} Explain 
  (patiently) to your friend how tsunami differ from regular waves.  
  Include in your explanation the differences in terms of tsunami formation, 
  wave length, wave height, and speed.

\item
  Discuss how the abiotic environmental parameters found in the
  sublittoral zone affect community structure. Be sure to account for
  depth gradients. Where do the abiotic factors have the greatest
  influence on community structure, and why?
  
\item
  Discuss the biotic factors that affect sublittoral community
  structure. How does this interact with each other to influence
  structure?

\item
  What is the infralittoral zone? What is the circalittoral zone? Does
  the bottom of the infralittoral zone always coincide with PAR?
  Why or why not?

\item
  What are two keystone species that affect community structure near the
  bottom of the kelp zone? Explain how these species influence diversity
  in the sublittoral.

\item
  Describe the long-term interactions between kelp and sea urchins and 
  associated alternate stable states. How do you think this would influence 
  patch dynamics and diversity at larger scales (say, along the western 
  coast of Washington, Oregon, California and Baja California?)

\item
  What is a trophic cascade? Describe a trophic cascade that is
  representative of rocky subtidal communities.
  
\end{enumerate}


\end{document}