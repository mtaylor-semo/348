%!TEX TS-program = lualatex
%!TEX encoding = UTF-8 Unicode

\documentclass[nofonts, letterpaper]{tufte-handout}

%\geometry{showframe} % display margins for debugging page layout

\usepackage{graphicx} % allow embedded images
  \setkeys{Gin}{width=\linewidth,totalheight=\textheight,keepaspectratio}
  \graphicspath{{img/}} % set of paths to search for images
  
\usepackage{fontspec}
  \setmainfont[Ligatures={Common,TeX},Numbers={Proportional}]{Linux Libertine O}
  \setsansfont{Linux Biolinum O}
\usepackage{microtype}
\usepackage{enumitem}
\usepackage{multicol} % multiple column layout facilities
%\usepackage{hyperref}
%\usepackage{fancyvrb} % extended verbatim environments
%  \fvset{fontsize=\normalsize}% default font size for fancy-verbatim environments

% Change the header to shift the title to the left side of the page. 
% Replaced \quad with \hfill.  See \plaintitle in tufte-common.def
{\fancyhead[RE,RO]{\scshape{\newlinetospace{\plaintitle}}\hfill\thepage}}

\makeatletter
% Paragraph indentation and separation for normal text
\renewcommand{\@tufte@reset@par}{%
  \setlength{\RaggedRightParindent}{1.0pc}%
  \setlength{\JustifyingParindent}{1.0pc}%
  \setlength{\parindent}{1pc}%
  \setlength{\parskip}{0pt}%
}
\@tufte@reset@par

% Paragraph indentation and separation for marginal text
\renewcommand{\@tufte@margin@par}{%
  \setlength{\RaggedRightParindent}{0pt}%
  \setlength{\JustifyingParindent}{0.5pc}%
  \setlength{\parindent}{0.5pc}%
  \setlength{\parskip}{0pt}%
}

\makeatother

% Set up the spacing using fontspec features
\renewcommand\allcapsspacing[1]{{\addfontfeatures{LetterSpace=15}#1}}
\renewcommand\smallcapsspacing[1]{{\addfontfeatures{LetterSpace=10}#1}}

\title{Study Guide 07}
\author{Deep-Sea Communities}

\date{} % without \date command, current date is supplied

\begin{document}

\maketitle	% this prints the handout title, author, and date

%\printclassoptions

\section{Vocabulary}
\marginnote{\textbf{Study:} pgs. 477--487.}
\vspace{-1\baselineskip}
\begin{multicols}{2}
whale fall \\
marine snow \\
chemosynthesis \\
endosymbiosis \\
hydrothermal vent \\
cold seep \\
sand \\
silt \\
clay \\
mud 
\end{multicols}

\section{Concepts}

\begin{enumerate}

\item
  What happens to the nutritional quality of marine snow as it descends
  from the epipelagic to the abyssal plain. Why?

\item
  What are the stages of succession that have been described for whale
  fall communities? Name and describe each stage, in the order of
  occurrence from first to last. Include approximately the length (time) of each stage.

\item
  Briefly explain how plate tectonics relates to the formation of
  hydrothermal vent communities. What underwater geological feature
  related to tectonics is associated with hydrothermal vents? In other words,
  where would you go look for vent communities?
  
\item 
  How does plate tectonics explain why
  hydrothermal vent communities are temporary (100s of years)?
  
\item
  What are the primary differences between hydrothermal vents and cold
  seeps. Consider energy sources, temperature, diversity, and endemism.

%\item
%  I argued that nutrient limitation drives most of the community
%  interactions in the deep-sea. What is the basis of my argument? How do
%  space and light fit into this argument?

\item
  About how much NPP is exported from hydrothermal vents and cold seeps?
  What are other sources of NPP to the deep-sea communities?

\item
  Although the amount of NPP leaving hydrothermal vent communities is
  minimal, some is exported to the larger deep sea habitat. Briefly
  describe the ways in which NPP can leave the vent communities.
  
\item
  Most chemosynthetic bacteria live in a fairly narrow area around the
  hydrothermal vents. Why is this area restricted in size?

\item
  Chemosynthetic bacteria can actually be found well away from the
  actual hot vents. Where would you find these bacteria? (I don't know
  if this answer is blindingly obvious or not.)

\item
  Diagram the distribution of biomass, density, and diversity from the 
  shelf break down to about 6000 m.
 
\item
  Explain how deep-sea benthic diversity can be high if both biomass and 
  density are low.

\item
  Describe the environment of a typical deep-sea benthic community 
  found on the abyssal plain well away from hydrothermal vents. Then,
  describe how the environment and nutrient availability determines the types
  of organisms that will be found int that community (where they live, how they feed, etc.).

\end{enumerate}



\end{document}