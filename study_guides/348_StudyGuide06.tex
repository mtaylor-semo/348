%!TEX TS-program = lualatex
%!TEX encoding = UTF-8 Unicode

\documentclass[nofonts, letterpaper]{tufte-handout}

%\geometry{showframe} % display margins for debugging page layout

\usepackage{graphicx} % allow embedded images
  \setkeys{Gin}{width=\linewidth,totalheight=\textheight,keepaspectratio}
  \graphicspath{{img/}} % set of paths to search for images
  
\usepackage{fontspec}
  \setmainfont[Ligatures={Common,TeX},Numbers={Proportional}]{Linux Libertine O}
  \setsansfont{Linux Biolinum O}
\usepackage{microtype}
\usepackage{enumitem}
\usepackage{multicol} % multiple column layout facilities
%\usepackage{hyperref}
%\usepackage{fancyvrb} % extended verbatim environments
%  \fvset{fontsize=\normalsize}% default font size for fancy-verbatim environments

% Change the header to shift the title to the left side of the page. 
% Replaced \quad with \hfill.  See \plaintitle in tufte-common.def
{\fancyhead[RE,RO]{\scshape{\newlinetospace{\plaintitle}}\hfill\thepage}}

\makeatletter
% Paragraph indentation and separation for normal text
\renewcommand{\@tufte@reset@par}{%
  \setlength{\RaggedRightParindent}{1.0pc}%
  \setlength{\JustifyingParindent}{1.0pc}%
  \setlength{\parindent}{1pc}%
  \setlength{\parskip}{0pt}%
}
\@tufte@reset@par

% Paragraph indentation and separation for marginal text
\renewcommand{\@tufte@margin@par}{%
  \setlength{\RaggedRightParindent}{0pt}%
  \setlength{\JustifyingParindent}{0.5pc}%
  \setlength{\parindent}{0.5pc}%
  \setlength{\parskip}{0pt}%
}

\makeatother

\title{Study Guide 06}
\author{The Deep Sea}

\date{} % without \date command, current date is supplied

\begin{document}

\maketitle	% this prints the handout title, author, and date

The “deep sea” refers broadly to the ocean below 200 meters.
%\printclassoptions

\section{Vocabulary}
\marginnote{\textbf{Study:} pgs. 27--29, 154--156 (bioluminescence, including the Hot Topic), 175--177.}
\vspace{-1\baselineskip}
\begin{multicols}{2}
mesopelagic \\
bathypelagic \\
disphotic \\
aphotic \\
thermocline \\
thermohaline circulation \\
Great Ocean Conveyor \\
bioluminescence \\
diel migration
\end{multicols}

\section{Concepts}

\begin{enumerate}

\item
  Distinguish between the mesopelagic, bathypelagic, disphotic, and aphotic zones. 
  explain why the mesophotic and disphotic zones are not \emph{exactly} the same thing.
  Do the same for the bathypelagic and aphotic zones.

\item
  Explain what is a thermocline? Tell why a thermocline prevents oxygen from 
  reaching the deep sea.
  
\item
  Describe thermohaline circulation.\marginnote{Remember that the salt does not freeze 
  	along with water when ice forms. The salt remains in the unfrozen water. } Tell 
  how temperature and salinity affects surface water density and causes the Great Ocean Conveyor.

\item
  How much does pressure change for every 10 meters of depth? Does the weight of the
  Earth's atmosphere above the surface of the ocean contribute to the pressure felt
  by organisms living underwater? Explain why or why not.

\item
  Know that turbidity (sediments, plankton, etc.) decreases light intensity below the surface.

\item
  Why are red and black color equally effective as camouflage in the deep sea?

\item
  Why do many mesopelagic fishes have laterally (sie-to-side) compressed bodies with
  countershading?
  
\item
	Why do aphotic (bathypelagic) fishes usually have blobby, jelly-like bodies with
	weak skeletons?
	
\item
	Describe adaptions that increase visual sensitivity in disphotic conditions. How
	do these adaptations differ in aphotic conditions?


\item
  	Why do many mesopelagic organisms have photophores lining their ventral side but
  	few or none on the dorsal side?
  	
\item
  	Describe the many ways in which bioluminescence can be used. Explain for
  	camouflage, predator-prey interactions, and intraspecific communication.
  	
\item
	Describe each of the many different feeding adaptations of deep-sea fishes. Explain
	why these adaptations are important in terms of \textsc{npp.}
  
\item
	Describe the many ways in which \textsc{npp} can be exported from the epipelagic to
	the deep sea.

 
 \end{enumerate}


\end{document}