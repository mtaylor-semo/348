%!TEX TS-program = lualatex
%!TEX encoding = UTF-8 Unicode

\documentclass[nofonts, letterpaper]{tufte-handout}

%\geometry{showframe} % display margins for debugging page layout

\usepackage{graphicx} % allow embedded images
  \setkeys{Gin}{width=\linewidth,totalheight=\textheight,keepaspectratio}
  \graphicspath{{img/}} % set of paths to search for images
  
\usepackage{fontspec}
  \setmainfont[Ligatures={Common,TeX},Numbers={Proportional}]{Linux Libertine O}
  \setsansfont{Linux Biolinum O}
\usepackage{microtype}
\usepackage{enumitem}
\usepackage{multicol} % multiple column layout facilities
\usepackage[version=4]{mhchem}
%\usepackage{hyperref}
%\usepackage{fancyvrb} % extended verbatim environments
%  \fvset{fontsize=\normalsize}% default font size for fancy-verbatim environments

% Change the header to shift the title to the left side of the page. 
% Replaced \quad with \hfill.  See \plaintitle in tufte-common.def
{\fancyhead[RE,RO]{\scshape{\newlinetospace{\plaintitle}}\hfill\thepage}}

\makeatletter
% Paragraph indentation and separation for normal text
\renewcommand{\@tufte@reset@par}{%
  \setlength{\RaggedRightParindent}{1.0pc}%
  \setlength{\JustifyingParindent}{1.0pc}%
  \setlength{\parindent}{1pc}%
  \setlength{\parskip}{0pt}%
}
\@tufte@reset@par

% Paragraph indentation and separation for marginal text
\renewcommand{\@tufte@margin@par}{%
  \setlength{\RaggedRightParindent}{0pt}%
  \setlength{\JustifyingParindent}{0.5pc}%
  \setlength{\parindent}{0.5pc}%
  \setlength{\parskip}{0pt}%
}

\makeatother

% Set up the spacing using fontspec features
\renewcommand\allcapsspacing[1]{{\addfontfeatures{LetterSpace=15}#1}}
\renewcommand\smallcapsspacing[1]{{\addfontfeatures{LetterSpace=10}#1}}

\title{Study Guide 14}%\hfill}
\author{Ocean Acidification: Cause}

\date{} % without \date command, current date is supplied

\begin{document}

\maketitle	% this prints the handout title, author, and date

%\printclassoptions

\section{Vocabulary}
\marginnote{\textbf{Study:} pgs. 32--27, 271--272, 448--450, 582--584, plus Hoegh-Guldberg et al. 2007, available on the course website.}
\vspace{-1\baselineskip}
\begin{multicols}{2}
ocean acidification \\
pH \\
bicarbonate buffering system \\
\ce{CO3^{2-}} (carbonate) \\
\ce{HCO3^{-}} (bicarbonate) \\
aragonite \\
$\Omega$ \\
rugosity \\
fecundity 
\end{multicols}

\section{Concepts}

\begin{enumerate}
\item Recognize and write equilibrium equation for the bicarbonate buffering system. Explain how it works to buffer pH.

\item Explain how addition of \ce{CO2} to the atmosphere increases ocean acidity.

\item In the bicarbonate buffering equation, identify the course of {H+} ions (protons) most responsible for ocean acidification.
 
\item What value of $\Omega$ indicates the water is saturated with aragonite? What is the optimum value of $\Omega$ for marine organisms that produce calcium carbonate structures?

\item Identify the source of carbonate for the bicarbonate buffering equation.

\item How have aragonite levels changed in the ocean since the 1700s. Explain how this change relates to pH change in the oceans.

\item Explain how the Paleocene-Eocene mass coral extinction provides evidence for a possible mass extinction of modern corals if pH continues to decrease.

\item Understand how to interpret the figure on climate change and ecological feedback from Hoegh-Guldberg et al. 2007.  I’m just saying\dots.

\item Explain how change in coral cover and grazing can change a reef between coral-dominated and algal-dominated states.


\end{enumerate}


\end{document}