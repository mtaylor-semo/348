%!TEX TS-program = lualatex
%!TEX encoding = UTF-8 Unicode

\documentclass[nofonts, letterpaper]{tufte-handout}

%\geometry{showframe} % display margins for debugging page layout

\usepackage{graphicx} % allow embedded images
  \setkeys{Gin}{width=\linewidth,totalheight=\textheight,keepaspectratio}
  \graphicspath{{img/}} % set of paths to search for images
  
\usepackage{fontspec}
  \setmainfont[Ligatures={Common,TeX},Numbers={Proportional}]{Linux Libertine O}
  \setsansfont{Linux Biolinum O}
\usepackage{microtype}
\usepackage{enumitem}
\usepackage{multicol} % multiple column layout facilities
%\usepackage{hyperref}
%\usepackage{fancyvrb} % extended verbatim environments
%  \fvset{fontsize=\normalsize}% default font size for fancy-verbatim environments

% Change the header to shift the title to the left side of the page. 
% Replaced \quad with \hfill.  See \plaintitle in tufte-common.def
{\fancyhead[RE,RO]{\scshape{\newlinetospace{\plaintitle}}\hfill\thepage}}

\makeatletter
% Paragraph indentation and separation for normal text
\renewcommand{\@tufte@reset@par}{%
  \setlength{\RaggedRightParindent}{1.0pc}%
  \setlength{\JustifyingParindent}{1.0pc}%
  \setlength{\parindent}{1pc}%
  \setlength{\parskip}{0pt}%
}
\@tufte@reset@par

% Paragraph indentation and separation for marginal text
\renewcommand{\@tufte@margin@par}{%
  \setlength{\RaggedRightParindent}{0pt}%
  \setlength{\JustifyingParindent}{0.5pc}%
  \setlength{\parindent}{0.5pc}%
  \setlength{\parskip}{0pt}%
}

\makeatother

% Set up the spacing using fontspec features
\renewcommand\allcapsspacing[1]{{\addfontfeatures{LetterSpace=15}#1}}
\renewcommand\smallcapsspacing[1]{{\addfontfeatures{LetterSpace=10}#1}}

\title{Study Guide 10}
\author{Coral Reefs}

\date{} % without \date command, current date is supplied

\begin{document}

\maketitle	% this prints the handout title, author, and date

%\printclassoptions

\section{Vocabulary}
\marginnote{\textbf{Read:} pgs. 432--455.\\\textbf{Questions:} pg. 459, 11--17, 20.}
\vspace{-1\baselineskip}
\begin{multicols}{2}
coral reef\\
dinoflagellates\\
zooxanthellae\\
fringing reef\\
barrier reef\\
atoll\\
reef flat\\
reef crest\\
fore-reef slope\\
%back-reef slope\\
lagoon\\
patch reef\\
exploitative competition\\
interference competition
\end{multicols}

\section{Concepts}

\begin{enumerate}

 \item
 What is a coral reef?  Where are they typically found? List
 the major physical factors that determine the global distribution
 of coral reefs (not zonation within reefs)? Relate each factor to the
 biology of corals to explain the distributional pattern.

\item
  Coral reefs are biogenic ecosystems. What is a biogenic ecosystem? 
  
\item
  How do coral reefs get larger?  What organisms contribute to the 
  building of reefs?  Explain how each contributes to the building of a reef.

\item
  Explain why coral reefs are not found near the outflow of large 
  rivers such as the Amazon River.

\item
  Coral reefs are often described as ``an oasis in a desert sea'' or 
  by a similar metaphor.  Explain.

\item
  Describe the model of coral reef formation developed by Charles
  Darwin.  Explain the modifications to this model necessary to 
  explain fringing and barrier reefs on continental margins.

\item
  Why are atolls found primarily in the Pacific and Indian oceans, 
  but nearly absent in the Atlantic Ocean?  (In the Atlantic, a few are
  found in the Caribbean.)

\item
  Compare and contrast the zonation of a reef as it develops from a 
  fringing reef to a barrier reef to an atoll.  What types of corals, in 
  terms of shape or structure, are typical of each zone?  What abiotic
  factors determine the distribution of these corals in the different zones?

\item
  Explain why coral reefs are among the most productive of any marine 
  community.  Consider both primary productivity and nutrient cycles.

\item
  What are the sources of GPP on coral reefs? If reefs are so highly
  productive, why does the water around the reefs not show signs of
  higher NPP exported from the reef? If the waters are nutrient-poor,
  what are the nutrient sources for reef GPP?

\item
  Why is coral reef diversity (not just corals themselves) highest in
  the Indo-West Pacific, particularly around Indonesia? Why is coral
  reef diversity lower in the Caribbean Sea? Offer both evolutionary and
  ecological explanations.

\item
  Draw a profile of a typical coral reef and label the zones. Explain
  the physical and biological factors that determine zonation on a coral
  reef. Which factors, physical or biological, best explain the pattern
  of zonation? Why?

\item
  For each zone, state whether competition, predation and
  positive-species interactions (e.g., mutualism) are high, moderate or
  low. Explain why for each.
  
\item
  Explain the difference between interference and exploitative
  competition. Which types of corals tend to show exploitative
  competition? Which types of corals tend to show interference
  competition? How is the balance between these two groups of species
  affected by corallivorous fishes?

\item
  Explain why coral reefs show enormous variation in community structure
  and assemblage in both space and time?
  
\item
  Explain each of the four hypotheses that have been proposed to explain the
  community structure of fishes on coral reefs.

\end{enumerate}

\end{document}