%!TEX TS-program = lualatex
%!TEX encoding = UTF-8 Unicode

\documentclass[nofonts, letterpaper]{tufte-handout}

%\geometry{showframe} % display margins for debugging page layout

\usepackage{graphicx} % allow embedded images
  \setkeys{Gin}{width=\linewidth,totalheight=\textheight,keepaspectratio}
  \graphicspath{{img/}} % set of paths to search for images
  
\usepackage{fontspec}
  \setmainfont[Ligatures={Common,TeX},Numbers={Proportional}]{Linux Libertine O}
  \setsansfont{Linux Biolinum O}
\usepackage{microtype}
\usepackage{enumitem}
\usepackage{multicol} % multiple column layout facilities
%\usepackage{hyperref}

\usepackage{chemfig}
\usepackage[version=4]{mhchem}
\usepackage{siunitx}

% Change the header to shift the title to the left side of the page. 
% Replaced \quad with \hfill.  See \plaintitle in tufte-common.def
{\fancyhead[RE,RO]{\scshape{\newlinetospace{\plaintitle}}\hfill\thepage}}

\makeatletter
% Paragraph indentation and separation for normal text
\renewcommand{\@tufte@reset@par}{%
  \setlength{\RaggedRightParindent}{1.0pc}%
  \setlength{\JustifyingParindent}{1.0pc}%
  \setlength{\parindent}{1pc}%
  \setlength{\parskip}{0pt}%
}
\@tufte@reset@par

% Paragraph indentation and separation for marginal text
\renewcommand{\@tufte@margin@par}{%
  \setlength{\RaggedRightParindent}{0pt}%
  \setlength{\JustifyingParindent}{0.5pc}%
  \setlength{\parindent}{0.5pc}%
  \setlength{\parskip}{0pt}%
}

\makeatother

% Set up the spacing using fontspec features
%\renewcommand\allcapsspacing[1]{{\addfontfeatures{LetterSpace=15}#1}}
%\renewcommand\smallcapsspacing[1]{{\addfontfeatures{LetterSpace=10}#1}}

% See https://tex.stackexchange.com/a/37711/39194
\renewcommand{\allcapsspacing}[1]{{\addfontfeature{LetterSpace=20.0}#1}}
\renewcommand{\smallcapsspacing}[1]{{\addfontfeature{LetterSpace=5.0}#1}}
\renewcommand{\textsc}[1]{\smallcapsspacing{\textsmallcaps{#1}}}
\renewcommand{\smallcaps}[1]{\smallcapsspacing{\scshape\MakeTextLowercase{#1}}}


\title{Study Guide 05}
\author{Primary Production and Ocean Ecosystem Models}

\date{} % without \date command, current date is supplied

\begin{document}

\maketitle	% this prints the handout title, author, and date

%\printclassoptions

\section{Vocabulary}
\marginnote{\textbf{Study:} pgs.~198--203, 208--211, 219--222, 225--226.}
\vspace{-1\baselineskip}
\begin{multicols}{2}
diatoms \\
Bacillariophyceae \\
dinoflagellates \\
Dinophyceae \\
primary production \\
gross primary production \\
net primary production \\
compensation depth \\
compensation intensity \\
\textsc{par} \\
Redfield ratio\\
bacterioplankton \\
microbial loop
\end{multicols}

\section{Concepts}

\begin{enumerate}

\item
  You\marginnote{Dissolved or particulate, organic or inorganic, nitrogen, phosphorus, etc.} should know the basic nutrient acronyms: \textsc{dic, doc, dip, dop, din, don, dom,} etc.

\item
  What is primary production? What is photosynthesis?\marginnote{\ce{CO2 + 6H2O ->[energy] C6H12O6 + 6O2}} Are they the same
  thing? Explain why or why not?

\item
  What factors most limit gross primary production by net
  phytoplankton? Which of these factors imposes the greatest limit to
  primary production? Why?

\item
  Which nutrients are most limiting to phytoplankton?

\item What is the Redfield ratio? What elements are included in the ratio?  What is the ratio of each element in phytoplankton and the surrounding water?  How does it show which nutrient(s) is/are limiting?

\item
  Explain the difference between gross and net primary production.
\item
  Explain the differences and similarities of compensation depth and
  compensation intensity.
\item
  Explain the factors that influence photosynthetically available
  radiation (\textsc{par}) at depth in the water column (for example, at 50\,m or
  75\,m of depth). Consider both the atmospheric conditions and oceanic
  conditions.
\item
  Considering (for now) only light intensity, why is primary
  productivity higher at roughly 20\,m depth than at the surface?


\item
  Summarize and illustrate the ``classic model'' of primary productivity in the ocean. Your
  summary should name the major phytoplankton and zooplankton organisms
  and describe their roles in overall net primary production. Your
  summary should also describe the relative availability of nutrients
  and light to explain the timing of peak productivity.
  
\item
  The classic model of primary production still works for several marine
  ecosystems. Which ones? However, the model is not satisfactory for
  other marine ecosystems. Name these ecosystems and state why the
  classic model doesn't fully explain gross primary production.
  
\item
  Discuss the limitations imposed by nutrients and copepod grazing on
  gross primary production in the classic model.
  
\item
  The classic ecosystem model of primary productivity accounts only for
  copepods, often only a single species of copepod. Yet, there are
  hundreds of species of net zooplankton that graze on phytoplankton. Why
  are these other net zooplankton not included in the model? (Think: grazing)

\item
  Similarly, the classic model accounts for primary production by
  diatoms and dinoflagellates, but there are hundreds of species of net
  phytoplankton in the epipelagic. Why are the other net phytoplankton
  not included in the model?
  
\item
  What is the microbial loop? What is the importance of the microbial
  loop to DOM (dissolved organic matter) in the epipelagic zone?
  
\item
  Illustrate a complete model for primary productivity and energy flow
  in the marine ecosystem that incorporates both the classic model and
  the updated model with the microbial loop. Include the major
  nutrients, and major groups of organisms involved.
  
\item
  Energy transfer from primary producers to primary consumers in marine
  ecosystems is more efficient than energy transfer between in the same
  trophic levels in terrestrial ecosystems. Explain why.
  
\item
  Total \textsc{npp} of the oceans accounts for nearly half (ca. 46\%) of total
  global \textsc{npp}, yet the per area average~\textsc{npp} (i.e., \unit{\kilo\gram\,C \per\square\meter\per\,yr}) is much lower for oceans than for
  terrestrial ecosystems. Account for this apparent discrepancy.
 
 \end{enumerate}


\end{document}