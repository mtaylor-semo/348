%!TEX TS-program = lualatex
%!TEX encoding = UTF-8 Unicode

\documentclass[t]{beamer}

%%%% HANDOUTS For online Uncomment the following four lines for handout
%\documentclass[t,handout]{beamer}  %Use this for handouts.
%\usepackage{handoutWithNotes}
%\pgfpagesuselayout{3 on 1 with notes}[letterpaper,border shrink=5mm]
%	\setbeamercolor{background canvas}{bg=black!5}


%%% Including only some slides for students.
%%% Uncomment the following line. For the slides,
%%% use the labels shown below the command.
%\includeonlylecture{student}

%% For students, use \lecture{student}{student}
%% For mine, use \lecture{instructor}{instructor}


%\usepackage{pgf,pgfpages}
%\pgfpagesuselayout{4 on 1}[letterpaper,border shrink=5mm]

% FONTS
\usepackage{fontspec}
\def\mainfont{Linux Biolinum O}
\setmainfont[Ligatures={Common,TeX}, Contextuals={NoAlternate}, BoldFont={* Bold}, ItalicFont={* Italic}, Numbers={Proportional}]{\mainfont}
\setsansfont[Scale=MatchLowercase]{Linux Biolinum O} 
\usepackage{microtype}

\usepackage{graphicx}
	\graphicspath{%
	{/Users/goby/Pictures/teach/348/5_in_5/}}%

\usepackage{amsmath,amssymb}

%\usepackage{units}

\usepackage{booktabs}
\usepackage{multicol}
%	\setlength{\columnsep=1em}

%\usepackage{textcomp}
%\usepackage{setspace}
\usepackage{tikz}
	\tikzstyle{every picture}+=[remember picture,overlay]

\definecolor{eveblue}{RGB}{80,152,242}

\mode<presentation>
{
  \usetheme{Lecture}
  \setbeamercovered{invisible}
  \setbeamertemplate{items}[square]
}

%\usepackage{calc}
%\usepackage{hyperref}

\newcommand\HiddenWord[1]{%
	\alt<handout>{\rule{\widthof{#1}}{\fboxrule}}{#1}%
}



\begin{document}

{
\usebackgroundtemplate{\includegraphics[width=\paperwidth]{reef_fish_james_watt_noaa}
}
\begin{frame}[b,plain]{\textcolor{white}{Larval dispersal and population connectivity.}}
\small{\textcolor{gray}{5-in-5 demo.}}\hfill \textcolor{white}{\tiny Photo by James Watt, NOAA, Creative Commons.}
\end{frame}
}

{
\usebackgroundtemplate{\includegraphics[width=\paperwidth]{roberts}
}
\begin{frame}[b,plain]
\tiny\textcolor{white}{Modified from Roberts 1997. Science 278: 1454}
\end{frame}
}

{
\usebackgroundtemplate{\includegraphics[width=\paperwidth]{cowen}
}
\begin{frame}[b,plain]
\tiny\textcolor{white}{Modified from Cowen et al. 2000. Science 287: 857}
\end{frame}
}


{
\usebackgroundtemplate{\includegraphics[width=\paperwidth]{evelynae.jpg}
}
\begin{frame}[b,plain]
\hfill \textcolor{white}{\tiny \textit{Elacatinus evelynae} \textcopyright\,Paul Humann, All Rights Reserved. Used by permission.}
\end{frame}
}


{
\usebackgroundtemplate{\includegraphics[width=\paperwidth]{eve_sample}
}
\begin{frame}[t,plain]
	\begin{tikzpicture}[form/.style={circle, inner sep=0pt, minimum size=2.5mm}]
		\node [white, right] at (8.75,-0.5) {Sharknose Goby};
		\node[form] (blueform) at (9,-1) [fill=eveblue] {};
		\node [eveblue,right] at (blueform.east) {blue form}; 
		\node[form] (yellowform) at (9,-1.5) [fill=yellow] {};
		\node [yellow,right] at (yellowform.east) {yellow form}; 
		\node[form] (whiteform) at (9,-2) [circle, fill=white] {};
		\node [white,right] at (whiteform.east) {white form}; 
		\node [white, right] at (-0.4, -9.3) {\tiny Distributions courtesy Pat Colin, unpublished data};
	\end{tikzpicture}
\end{frame}
}

{
\usebackgroundtemplate{\includegraphics[width=\paperwidth]{eve_cytb_network}
}
\begin{frame}[t,plain]
	\begin{tikzpicture}
		\node [white, right] at (8.5,-0.5) {Sharknose Goby};
		\node [orange5, right] at (8.5,-1) {\small $\Phi_{\rm{ST}} = 79\%$};
		\node [white, right] at (8.5,-1.5) {\scriptsize cytochrome \textit{b} (400 bp)};
		\node [white,right] at (-0.4,-9.3) {\tiny Taylor and Hellberg 2003. Science 299: 107};
	\end{tikzpicture}

\end{frame}
}

\end{document}
